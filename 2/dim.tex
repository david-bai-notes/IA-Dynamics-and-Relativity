\section{Dimensional Analysis}
\subsection{Basic Dimensional Quantities and Units}
For most motions we will be considering, there are basically three dimensions of interests: length ($L$), mass ($M$) and time ($T$).
In general, the dimension of a physical quantity $X$ can be expressed using thsese three dimensions.
For example, the density can be expressed by $ML^{-3}$, and the force can be expressed by $MLT^{-2}$.
We are only going to consider the product of powers of the dimensional quantities.
We can then introduce units for the basic dimensional quantities.
Most likely we will use the SI unit system ($L=$m, $M=$kg, $T=$s).
For other quantities, we can form units out of the basic units we defined for the basic quantities.
\begin{example}
    To find the unit of the constant $G$ in Newton's Law of Gravitation, we can determine by writing each quantity in basic quantities, so we have $G=L^3T^{-2}M^{-1}$, therefore the unit for $G$ will be ${\rm m^3s^{-2}kg^{-1}}$.
\end{example}
The general principle is that dynamical or physical equations must work for any chosen system of units.
\subsection{Scaling}
Suppose I have a dimensional quantity $Y$ which depends on some other quantities $X_1,X_2,\ldots,X_n$.
Let the diensions of the quantity $Y$ be $L^aM^bT^c$, and $X_i$ has dimensions $L^{a_i}M^{b_i}T^{c_i}$.
We want to determine the dimensions of $Y$ from that of $X_i$.
So obviously we have $Y=C\prod_i X_i^{p_i}$, so
$$\begin{cases}
    a=\sum_ip_ia_i\\
    b=\sum_ip_ib_i\\
    c=\sum_ip_ic_i
\end{cases}$$
If $n=3$, then there is an unique solution iff $X_1,X_2,X_3$ are independent, so
$$\begin{vmatrix}
    a_1&a_2&a_3\\
    b_1&b_2&b_3\\
    c_1&c_2&c_3
\end{vmatrix}\neq 0$$
which happens most of the time.
Note that for general $n$ there must be a solution (not necessarily unique) if we assume that $Y$ does indeed depend on a subset of $\{X_i\}$.
So for $n<3$, there is an unique solution as well.\\
For $n>3$, however, we can choose $n-3$ dimensionless constants
$$\lambda_i=X_i/(X_1^{p_{i1}}X_2^{p_{i2}}X_3^{p_{i3}})$$
where $i=4,5,\ldots$, assuming $X_1,X_2,X_3$ are independent.
So
$$Y=C(\lambda_4.\lambda_5,\ldots)X_1^{p_1}X_2^{p_2}X_3^{p_3}$$
where $C$ is a dimensionless function.
This is sometimes known as Bridgemzn's Theorem.
\begin{example}
    Consider a simple pendulum.
    Let $d$ be the horizontal initial displacement, $m$ the mass, and $g$ the acceleration due to gravity, and $l$ the length of the string, and we want to find expression of the period $P$ in term of these.
    Speaking of dimensions,
    $$\begin{cases}
        [P]=T\\
        [d,l]=L\\
        [g]=LT^{-2}\\
        [m]=M
    \end{cases}$$
    Thus
    $$T=M^{p_1}L^{p_2}(LT^{-2})^{p_3}$$
    solve to get $p_1=0,p_2=1/2,p_3=-1/2$
    Hence
    $$P=C\left( \frac{d}{l} \right)\sqrt{\frac{l}{g}}$$
    For a dimensionless function $C$.\\
    Hence, if we scale $d,l$ by $2$, the period will be scaled by $\sqrt{2}$.
    Also $P$ is independent of $m$.
\end{example}
\begin{example}
    Taylor's estimate to the first atomic explosion.\\
    We want to estimate the radius of the fireball $R$ which has dimension $L$.
    $R$ depends on the time $t$ since the explosion which has dimension $T$.
    The density of air $\rho_0$ which has dimension $ML^{-3}$ is also involved.
    Lastly the energy of explosion $E$ having dimension $ML^2T^{-2}$.\\
    So by doing dimensional analysis, we immediately (since there are only $3$ depending dimensions) have $R\propto \sqrt[5]{Et^2/\rho_0}$.
    This has allowed Taylor to estimate the size of $E$.
\end{example}