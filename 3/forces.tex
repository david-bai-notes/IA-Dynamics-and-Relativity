\section{Forces}
\subsection{Force and Potential Energy in One (Spacial) Dimension}
Consider a point mass $m$ moving on a straight line with position given by $x(t)$.
We assume that the force $F=F(x)$ depends entirely on position, not velocity and time.
\begin{definition}
    The potential energy $V(x)$ is any function that satisfies $F(x)=-\mathrm dV/dx$.
\end{definition}
Equivalently,
$$V(x)=V(0)+\int_0^xF(x)\,\mathrm dx$$
where $V(0)$ can be taken arbitrarily.
The equation of motion is simply $m\ddot{x}=-\mathrm dV/dx$ by Newton's Second Law.
\begin{definition}
    The KInetic energy $T$ is defined by $T=m|\dot{x}|^2/2$
\end{definition}
\begin{theorem}
    Under the assumptions and definitions above, we have $\mathrm d(T+V)/\mathrm dt=0$.
\end{theorem}
\begin{proof}
    $$\frac{\mathrm d(T+V)}{\mathrm dt}=\frac{2m\dot{x}\ddot{x}}{2}+\frac{\mathrm dV}{\mathrm dx}\frac{\mathrm dx}{\mathrm dt}=\dot{x}(m\ddot{x}+\frac{\mathrm dV}{\mathrm dx})=0$$
    By Newton's Second Law.
\end{proof}
Note that if we lost that restriction on the time and velocity independence of the force, we lose the conservation of energy in general.
\begin{example}
    Consider a harmonic oscillator, so $F(x)=-kx$ where $k$ is a positive constant.
    So $V(x)=kx^2/2$ by choosing the arbitrary constant as $0$.
    We want to calculate all the stuff to verify the conservation of energy.
    \footnote{Hey, you literally just proved it.}
    We can solve the motion by solving $m\ddot{x}=-kx$ which solves to $x=A\sin(\sqrt{k/m}t)+B\cos(\sqrt{k/m}t)$.
    And plugging in gives $\mathrm dE/\mathrm dt=0$.
\end{example}
As the first instance of Newton's Second Law, the conservation of energy is a useful rule to determine a one dimensional motion.
Using conservation of energy, we have
$$\dot{x}=\pm\sqrt{\frac{2}{m}(E-V(x))}$$
which is a first order ODE.
So
$$\int_{x_0}^x\frac{\mathrm du}{\sqrt{2(E-V(u))/m}}=t-t_0$$
where $x(t_0)=x_0$.
In principle we can solve it to obtain the motion.\\
We can also have some qualitative insight from conservation of energy.
Consider $V(x)=\lambda(x^3-3\beta^2x)$ where $\lambda,\beta>0$ are constants.
We can sketch the potential energy to find that $V$ has a local maximum at $-\beta$, which value happens again at $x=2\beta$.
And it has a local minimum at $\beta$, where it again obtain the same value at $-2\beta$.
So we can find certain properties of the motion from the graph if the motion start at rest, then we must have $V(x)\le V(x_0)$\\
Case 1: $x_0<-\beta$, it will moves to left so as to reduce the potential and gain speed.\\
Case 2: $-\beta<x_0<2\beta$, then the particle will be restricted in the region $-\beta<x_0<2\beta$ and will oscillate.\\
Case 3: $x_0>2\beta$, then it will move to the right.\\
The case becomes special if we turn to the stationary (or equilibrium) points.
Obviously $x_0=-\beta$ is an unstable fixed point and $x_0=\beta$ is a stable fixed points.
So at $x_0=2\beta$, it will end its notion at the fixed point $x=-\beta$.
In this case, we can analyse the behaviour by writing down the integral.
This can show that the time to reach $x=-\beta$ is infinite when we approach to $2\beta$.
\subsection{Equilibriums}
The points $x=\pm\beta$ in this case are called equilibrium points, at which the particle can always stay at rest.
The condition for this to happen is $V^\prime(x_0)=0$.
We are going to analyze the motion near the equilibrium at $x_0$ by expanding its Taylor series
$$V(x)\approx V(x_0)+(x-x_0)V^\prime(x_0)+\frac{(x-x_0)^2}{2}V^{\prime\prime}(x_0)=V(x_0)+\frac{(x-x_0)^2}{2}V^{\prime\prime}(x_0)$$
We assume for a moment that $V^{\prime\prime}(x_0)$ does not vanish.
\footnote{If it does vanish, we will have to look at higher order terms.}
So we have $m\ddot{x}=-(x-x_0)V^{\prime\prime}(x_0)$.\\
If $V^{\prime\prime}(x_0)>0$, it's a minimum of $V$ which produces the equation of a harmonic oscillator with period $\sqrt{V^{\prime\prime}(x_0)/m}$.
In this case, we say it is a stable equilibrium.\\
If $V^{\prime\prime}(x_0)<0$, it's a maximum of $V$ which produces the equation of an exponentially growing solution.
Hence it is an unstable equilibrium with growth rate $\sqrt{-V^{\prime\prime}(x_0)/m}$.
\begin{example}
    We look back to a pendulum with mass $m$, length $l$ and angle $\theta$.
    If we think of Newton's Second Law, one can obtain
    $$F=ml\ddot{\theta}=-mg\sin\theta=-\frac{\mathrm d}{\mathrm d\theta}(-mg\cos\theta)$$
    So we have $E=T+V=ml^2\dot{\theta}^2/2-mgl\cos\theta$.
    One can check that $\dot{E}=0$.
    Also the potential $V(\theta)=-mg\cos\theta$ has stable equilibrium at $\theta=2\pi k,k\in\mathbb Z$ and unstable at $\theta=\pi+2\pi k,k\in\mathbb Z$.
    So if the initial value of $\theta$ is in $(-\pi,\pi)$ (or $|V|<mgl$), the pendulum will oscillate.
    If $|V|>mgl$, then it will go round and round.\\
    Now we want to analyze the period of oscillations.
    Suppose the original angle is at $\theta_0\in (0,\pi)$, then the oscillation is going to be $\theta_0\to 0\to-\theta_0\to0\to\theta_0$, so the period is $4$ times the time taken for $\theta_0$ to $0$, hence
    $$P=4\int_0^{\theta_0}\frac{\mathrm d\theta}{\sqrt{2gl(\cos\theta-\cos\theta_0)/l^2}}=4\sqrt{\frac{l}{g}}\int_0^{\theta_0}\frac{\mathrm d\theta}{\sqrt{2\cos\theta-2\cos\theta_0}}=\sqrt{\frac{l}{g}}F(\theta_0)$$
    For small $\theta_0$ we have
    $$F(\theta_0)\approx 4\int_0^{\theta_0}\frac{\mathrm d\theta}{\sqrt{\theta_0^2-\theta^2}}=2\pi$$
    Hence $P\approx 2\pi\sqrt{l/g}$.
\end{example}
\subsection{Force and Potential in Three Dimensions}
Consider a particle $\underline{r}$ in motion in three dimensional space.
Then $m\underline{\ddot{r}}$ and $T=m|\underline{\dot{r}}|^2/2$.
And the rate of change of $T$ is then
$$\frac{\mathrm dT}{\mathrm dt}=m\underline{\dot{r}}\cdot\underline{\ddot{r}}=\underline{\dot{r}}\cdot\underline{F}$$
Suppose the particle tranverse a path $C$ from $t_0$ to $t_1$, then
\begin{definition}
    The work done is
    $$\int_{t_0}^{t_1}\underline{F}\cdot\underline{\dot{r}}\,\mathrm dt=\int_C\underline{F}\cdot\mathrm d\underline{r}$$
\end{definition}
We can also write that the total work equals
$$\int_C\underline{F}\cdot\mathrm d\underline{r}=\int_C F_x\,\mathrm dx+F_y\,\mathrm dy+F_z\,\mathrm dz$$
Suppose that the force is a function of the position $\underline{F}(\underline{r})$ (also called a force field).
\begin{definition}
    A force field $\underline{F}(\underline{r})$ is called conservative if $\underline{F}=\nabla V$ for some $V:\mathbb R^3\to\mathbb R$.
\end{definition}
If a force field $\underline{F}(\underline{r})$ is conservative, we say $V$ is the potential function.
Also in this case $E=T+V(\underline{r})$ conserved.
Indeed,
$$\frac{\mathrm dE}{\mathrm dt}=\frac{\mathrm dT}{\mathrm dt}+\frac{\mathrm dV}{\mathrm dt}=m\underline{\dot{r}}\cdot\underline{\ddot{r}}+\nabla V\cdot\underline{\dot{r}}=\underline{\dot{r}}\cdot(m\underline{\ddot{r}}-\underline{F})=0$$
The total work done by a conservative force $\underline{F}$ is
$$\int_C\underline{F}\cdot\mathrm d\underline{r}=\int_C-\nabla V\cdot\mathrm d\underline{r}=V(\underline{r}(t_0))-V(\underline{r}(t_1))$$
So the work done is independent of the path taken.\\
In particular, if the curve is closed, no work is done.\\
$\underline{F}$ is conservative if $\nabla\times\underline{F}=\underline{0}$ (given that the domain is simply connected).
\subsection{Angular Momentum}
\begin{definition}
    The angular momentum for a particle with mass $m$ and velocity $\underline{\dot{r}}$ is defined as
    $$\underline{L}=\underline{r}\times\underline{p}=m\underline{r}\times\underline{\dot{r}}$$
    And
    $$\underline{G}=\frac{\mathrm d\underline{L}}{\mathrm dt}=m\underline{\dot{r}}\times\underline{\dot{r}}+m\underline{r}\times\underline{\ddot{r}}=\underline{r}\times\underline{F}$$
    is defined as the torque, or moment of force.
\end{definition}
Note that $\underline{L},\underline{G}$ both depend on the choice of origin, so we must specify them when talking about angular stuff.
\begin{remark}
    If $\underline{r}\times\underline{F}=\underline{0}$ then $\underline{G}=0$, thus $\underline{L}$ is constant.
    In this case, we say the angular momentum is conserved.
\end{remark}
\subsection{Central Forces}
A special type of conservative force occurs when the potential $V$ depends entirely on $|\underline{r}|$, so $V(\underline{r})=V(|\underline{r}|)=V(r)$,
\footnote{It's just a tiny abuse of notation. vErY tInY.}
$$\underline{F}(\underline{r})=-\nabla V(|\underline{r}|)=-\frac{\mathrm dV}{\mathrm dr}\underline{\hat{r}},\underline{\hat{r}}=\frac{\underline{r}}{|\underline{r}|}$$
So $\underline{F}$ and $\underline{r}$ are parallel, therefore $\mathrm d\underline{L}/\mathrm dt=\underline{G}=\underline{F}\times\underline{r}=0$.
\subsection{Gravity}
Recall that Newton's Gravitational Law states
$$V=-\frac{GMm}{|\underline{r}|},\underline{F}=-\nabla V=-\frac{GMm}{|\underline{r}|^2}\underline{\hat{r}}=-\frac{GMm}{|\underline{r}|^3}\underline{r}$$
Note that the $m$ here can be ignored (in the way shown below) if we are only interested in the motion due to Newton's Second law:
\begin{definition}
    The gravitational potential is defined by $\Phi_g(\underline{r})=V/M=-GM/|\underline{r}|$, and the gravitational field by $\underline{g}=-\nabla\Phi_g(\underline{r})=-GM\underline{\hat{r}}/r^2$.
\end{definition}
The gravitational field and its potential, as functions, are dependent of $M$ alone.
We also have $m\Phi_g=V,m\underline{g}=\underline{F}$.
For a set of more than one masses, we can simple generalize by adding the corresponding fields and potential together by the superposition principle.
Hence for continuous bodies, we can replace the sum by an integral.
In particular, if the body is spherical with radius $R$ and we have $|\underline{r}|>R$, then we do have $\Phi_g(\underline{r})=-GM/|\underline{r}|$, thus spherical bodies do behave like a point when measuring from above its surface.
\begin{note}
    The mass $m$ in Newton's Second Law $m\underline{\ddot{r}}=\underline{F}$ is called the inertial mass, whilist the mass in Newton's Gravitational Law $\underline{F}=-GMm\underline{r}/|\underline{r}|^2$ is the gravitational mass.
    These two definitions of mass are different in relativity but are very closely related (about a difference of $10^{-12}$).
    The precise difference will be discussed in General Relativity.
\end{note}
There are a few results about the effect of gravity.
\begin{example}[Potential Energy near the Surface]
    For a mass $m$ at height $z$ above a spherical mass $M$ with radius $R$, if $z<<R$, then the potential energy is given by
    \begin{align*}
        V(R+z)&=-\frac{GMm}{R+z}\\
        &=-\frac{GMm}{R}+\frac{GMm}{R^2}z+o(R^{-2})\\
        &\approx -\frac{GMm}{R}+mgz\\
        &=\text{const}+mgz
    \end{align*}
    For earth, we have the approximation $g\approx 9.8{\rm ms^{-2}}$
\end{example}
\begin{example}[Escape Velocity]
    We want to find the critical velocity $\underline{v}$, perpendicular to $\underline{r}$, to leave a planet.
    Due to the conservation of energy $E=T+V=m|\underline{v}|^2/2-GMm/R$, the particle can escape (i.e. $\underline{v}$ is nonnegative at infinity) iff the initial energy has $E_0\ge 0$, which happens iff
    $$m|\underline{v}|^2/2\ge GMm/R\implies |\underline{v}|\ge \sqrt{\frac{2GM}{R}}=v_{\rm esc}$$
\end{example}
\subsection{Electromagnetic Forces}
We have seen previously that for a point charge $q$, the force has the expression $\underline{F}=q(\underline{E}+\underline{\dot{r}}\times\underline{B})$
In general $\underline{E},\underline{B}$ are functions of $\underline{r}$ and $\underline{t}$.
These are known as the Lorentz Force Law.
For convenience or something, we are going to restrict ourselves to time-independent fields.
So we want the electric field to be conservative, i.e. $\underline{E}=-\nabla\Phi_e$ where $\Phi_e$ is called the electrostatic potential.
\begin{claim}
    In a time-independent electromagnetic field, the energy
    $$E=T+V=\frac{m|\underline{\dot{r}}|^2}{2}+q\Phi_e(\underline{r})$$
    is conserved.
\end{claim}
\begin{proof}
    \begin{align*}
        \frac{\mathrm dE}{\mathrm dt}&=\frac{\mathrm d}{\mathrm dt}\left(\frac{m|\underline{\dot{r}}|^2}{2}+q\Phi_e(\underline{r})\right)\\
        &=m\underline{\dot{r}}\cdot\underline{\ddot{r}}-q\underline{\dot{r}}\cdot\underline{E}\\
        &=\underline{\dot{r}}(m\underline{\ddot{r}}-q\underline{E})\\
        &=0
    \end{align*}
    So $E$ is constant.
\end{proof}
\begin{law}
    Now consider a point charge $Q$ located at the origin.
    It generates an electrostatic field
    $$\Phi_e(\underline{r})=\frac{Q}{4\pi\epsilon_0|\underline{r}|},\underline{E}=\frac{Q}{4\pi\epsilon_0|\underline{r}|^2}\underline{\hat{r}}$$
    where $\epsilon_0$ is called the electric constant.
\end{law}
So the force exerted on our point charge $q$ is
$$\underline{F}=q\underline{E}=\frac{Qq}{4\pi\epsilon_0|\underline{r}|^2}\underline{\hat{r}}$$
which is called the Coulomb force.
One observe the similarity of this with the gravitational law (inverse-square law).
Also, by considering the signs, we find that same signed charges repel, opposite charges attract.
\subsection{Friction}
The friction is a contact force, which occurs when two body touches each other (they may not be of the same form though).
It is a convenient description of complicated molecular-scale physics.
So friction is not a kind of fundamental forces (gravity, EM, strong force, weak force).\\
We first consider a special kind of friction that is \textit{dry friction}.
Solids stand on each other exerts an action and reaction pair of normal forces (normal to the surface of contact), which prevents an object from merging with the other.
There is also a tangential force, which is a tangent to the trajectory on the surface where the solid is moving.\\
Now imagine we place a block on a slope.
If it remains at rest, the tangential force is called the static force, which exists even without relative motion.
In this case we have
\begin{law}
    The static force $\underline{F}$ has $|\underline{F}|\le\mu_s|\underline{N}|$ where $\mu_s$ is a constant (depending on the materials) called the coefficient of static friction, and $\underline{N}$ is the normal force.
\end{law}
So the block can rest on the plane provided that $\alpha\le\tan^{-1}(\mu_s)$ where $\alpha$ is the inclination.\\
There is a kinetic frictional force as well, which depends on the kinetic motion of the object.
\begin{law}
    The kinetic frictional force has $\underline{F}=\mu_k|\underline{N}|$ where $\mu_k$ is also a constant depending on the materials.
\end{law}
Normally $\mu_s>\mu_k>0$.\\
The most complicated type of friction is the \textit{fluid drag}, which is the friction exerted by a solid moving in fluid medium.
The model of linear drag says that
$$\underline{F}=-k_1\underline{\underline{u}}$$
where $\underline{u}$ is the velocity along the direction of motion and $k$ is a constant.
This model is relevent if we are considering a small object moving through a viscous fluid.
\begin{law}[Stokes' Law]
    Consider a sphere with radius $R$ moving in a viscous fluid with viscosity of $\eta$, then we have $k_1=6\pi\eta R$.
\end{law}
Another drag regime is called the quadratic model, which is for large bodies moving in less viscous fluid.
$$\underline{F}=-k_2|\underline{u}|\underline{u}$$
Typically we have $\rho R^2C_0$ where $\rho$ is the density of the fluid, $R$ is the radius and $C_0$ is the drag coefficient.\\
In the case of a linear drag, the rate of work done is $\underline{F}\cdot\underline{u}=-k_1|\underline{u}|^2$ and for quadratic law $\underline{F}\cdot\underline{u}=-k_2|\underline{u}|^3$.
The fluid gains energy due to this energy loss by the solid, obviously.\\
Recall from differential equations that the damp oscillator $m\ddot{x}=-kx-\lambda\dot{x}$ where the last term is a drag.
We know how to solve this.
\begin{example}
    Projectiles moving under uniform gravity and experiencing linear drag force.
    The equation of motion is hence
    $$m\ddot{x}=m\underline{g}-k\underline{\dot{x}}$$
    Consider the particle start at the origin with some velocity $\underline{U}$, so the initial conditions are $\underline{x}(0)=\underline{0},\underline{\dot{x}}(0)=\underline{U}$.
    We can solve the equation in $\underline{\dot{x}}$ and substituting the initial condition gives
    $$\underline{\dot{x}}=\frac{m\underline{g}}{k}+(\underline{U}-m\underline{g}/k)e^{-kt/m}$$
    Integrate it again and plug in the other initial condition,
    $$\underline{x}=\frac{m\underline{g}}{k}t+\frac{m}{k}(\underline{U}-m\underline{g}/k)(1-e^{-kt/m})$$
    Set $\underline{x}=(x,y,z),\underline{U}=U(\cos\theta,0,\sin\theta),\underline{g}=(0,0,-g)$.
    Hence we have, by simply plugging things in, that $y$ direction is irrelevant at all, while
    $$\dot{x}=U\cos\theta e^{-kt/m},\dot{y}=0,\dot{z}=(U\sin\theta+mg/k)e^{-kt/m}-mg/k$$
    so the $x$-velocity will eventually go to $0$ and $z$ velocity to a terminal value $mg/k$ (roughly after $t=m/k$).\\
    As for displacement, we have
    $$x=mU\cos\theta/k(1-e^{-kt/m}),z=-mgt/k+m/k(U\sin\theta+mg/k)(1-e^{-kt/m})$$
    so $x$ is bounded but $z$ is eventually moving (linearly as $t$ large).\\
    Now we want to turn to analyze the range $R(U,\theta,m,k,g)$ of the projectile for it to reach its original position (assuming we project it upwards), so by dimensional analysis we get the dimensionless quantity to be $f(\theta,kU/mg)=f(\theta,(U/g)/(m/k))$.
    Note that $U/g$ is proportional to the time taken to reduce velocity such that it vanishes; and $m/k$ is the approximate time to achieve the terminal velocity.
    So weak friction means $kU/mg<<1$ and strong firction means $kU/mg>>1$.\\
    We have $R=U^2/gf(\theta,kU/mg)$.
    If $kU/mg<<1$, then $R\approx U^2/g(2\sin\theta\cos\theta)$.
    If $kU/mg>>1$, then $R\approx U^2/g(\cos\theta(mg/kU))$.
\end{example}