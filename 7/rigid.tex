\section{Rigid Bodies}
\begin{definition}
    A rigit body is an extended mass with a finite volume as a system of particles that are constrained such that the mutual distances between them does not change.
\end{definition}
\begin{definition}
    An isometry is a distance-preserving map in the space, e.g. rotation, translation, etc..
\end{definition}
So a rigid body is a system of particle moving under isometries.
\subsection{Angular Velocity}
Recall that we can have a vector angular velocity $\underline{\omega}$ which points to the axis of rotation and has magnitude equal to the scalar angular velocity $\omega$ of the point mass $\underline{r}$.
So $\underline{\dot{r}}=\underline{\omega}\times\underline{r}$.
If the particle has mass $m$, we can write down the kinetic energy $T=m|\underline{\dot{r}}|^2/2=m\omega^2r_\perp^2/2$ where $r_\perp=|\underline{n}\times\underline{r}|$ where $\underline{n}$ is a unit vector and $\underline{\omega}=\omega\underline{n}$.
We write $I=mr_\perp^2$ as the moment of inertia, so $T=I\omega^2/2$.
Note that the moment of inertia is dependent on the axis.
\subsection{Moment of Inertia for a Rigid Body}
Consider a rigid body made up of $N$ particles following the notation we introduced earlier.
The body (i.e. all the particles within it) would rotate about an axis through the origin with angular velocity $\underline{\omega}$.
For particle $i$, we have $\underline{\dot{r}_i}=\underline{\omega}\times\underline{r_i}$.
Note that
$$\frac{\mathrm d}{\mathrm dt}|\underline{r_i}-\underline{r_j}|^2=2((\omega\times(\underline{r_i}-\underline{r_j}))\cdot(\underline{r_i}-\underline{r_j}))=0$$
So the particles do stay the same distance apart.
The kinetic energy of the rotating body is then going to be
$$T=\sum_{i=1}^N\frac{1}{2}m_i|\underline{\dot{r}_i}|^2=\frac{1}{2}\omega^2\sum_{i=1}^Nm_i|\underline{n}\times\underline{r_i}|^2=\frac{1}{2}\omega^2\sum_{i=1}^Nm_i(r_i)_\perp^2=\frac{1}{2}\omega^2I$$
where $I$ is called the moment of inertia for the body.
Correspondingly, we can consider the angular momentum, where
$$\underline{L}=\sum_{i=1}^N\underline{L_i}=\sum_{i=1}^Nm_i\underline{r_i}\times(\underline{\omega}\times\underline{r_i})=\omega\sum_{i=1}^Nm_i\underline{r_i}\times(\underline{n}\times\underline{r_i})$$
Consider its component in the direction of the axis of rotation,
$$\underline{L}\cdot\underline{n}=\omega\sum_{i=1}^Nm_i\underline{n}\cdot(\underline{r_i}\times(\underline{n}\times\underline{r_i}))=\omega\sum_{i=1}^nm_i|\underline{n}\times\underline{r_i}|^2=I\omega$$
So the direction of $\underline{L}$ in the direction of the axis of rotation is $I\omega$.
In general, $\underline{L}$ is not parallel to $\underline{\omega}$, so we need to go back to the vector expression.
Observe that $\underline{L}$ as a function of $\omega$ is linear, so
$$\underline{L}=\sum_{i=1}^Nm_i\underline{r_i}\times(\underline{\omega}\times\underline{r_i})=\sum_{i=1}^Nm_i(|\underline{r_i}|^2\underline{\omega}-|\underline{r_i}\cdot\underline{\omega}|\underline{r_i})=I\underline{\omega}$$
where $I$ here is a tensor (i.e. a (multi)linear map) which in this case is a $3\times 3$ matrix.
So under suffix notation, $L_\alpha=I_{\alpha\beta}\omega_\beta$.
$I$ is a symmetric tensor (matrix) by symmetry.
We have
$$I_{\alpha\beta}=\sum_{i=1}^Nm_i(|\underline{r_i}|^2\delta_{\alpha\beta})-(\underline{r_i})_\alpha(\underline{r_i})_\beta$$
Now $I$ is diagonalizable so we can choose our favourite basis (principal axes) to make $I$ diagonal.
To get $\underline{L}$ to be at the same direction as $\underline{\omega}$, we need the object to rotate wrt a principal axis
\subsection{Calculation of Moment of Inertia}
For a solid body, we replace mass-weighted sums by mass-weighted volume integrals.
Consider a body with volume $V$ with density $\rho(\underline{r})$, so its mass, center of mass and moment of inertia are
$$M=\int_V\rho\,\mathrm dV,\underline{R}=\frac{1}{M}\int_V\rho(\underline{r})\underline{r}\,\mathrm dV,I=\int_V\rho(\underline{r})|\underline{r}_\perp|^2\,\mathrm dV=\int_V\rho(\underline{r})|\underline{n}\times\underline{r}|^2\,\mathrm dV$$
For curves and surfaces, we can use line and area integrals accordingly.
\begin{example}
    1. For uniform thin ring of mass $M$ and radius $a$ with rotation axis $\underline{n}$ through the center of the ring and perpendicular to the plane where the ring is on.
    In this case, we can reduce volume integral to line integral.
    We have $\rho=M/(2\pi a)$, so
    $$I=\int_0^{2\pi}\left( \frac{M}{2\pi a} \right)a^2a\,\mathrm d\theta=Ma^2$$
    Every point in the body is of the same distance from the axis $|\underline{r}_\perp|=|\underline{n}\times\underline{r}|=a$.\\
    2. Consider a uniform thin rod of mass $M$ and length $l$ with axis of rotation through one end and perpendicular to the rod.
    So
    $$I=\int_0^l\left( \frac{M}{l} \right)x^2\,\mathrm dx=\frac{1}{3}Ml^2$$
    3. Consider a uniform thin disk with mass $M$ and radius $a$ with the axis of rotation through its center and perpendicular to the plane where the disk is in.
    So we use an area integral
    $$I=\int_0^a\int_0^{2\pi}\left( \frac{M}{\pi a^2} \right)r^2r\,\mathrm d\theta\,\mathrm dr=\frac{Ma^2}{2}$$
    4. Using the same disk but choose the axis to be one through the center and in the same plane as the disk.
    In this case,
    $$I=\int_0^a\int_0^{2\pi}\left( \frac{M}{\pi a^2} \right)(r^2\sin^2\theta)r\,\mathrm d\theta\,\mathrm dr=\frac{1}{4}Ma^2$$
    5. Consider a solid sphere (a ball) of mass $M$ and radius $a$ with axis of rotation through its center, so spherical polars will be a good choice.
    We assume WLOG that $\underline{n}$ is the $z$ direction (so $\theta=0$ along $\underline{n}$).
    By uniform density, we have $\rho=3M/(4\pi a^3)$, therefore
    $$I=\int_0^a\int_0^\pi\int_0^{2\pi}\frac{3M}{4\pi a^3}(r^2\sin^2\theta)r^2\sin\theta\,\mathrm d\phi\,\mathrm d\theta\,\mathrm dr=\frac{2}{5}Ma^2$$
\end{example}
There are a few simple but general results to simplify calculation moment of inertia.
\begin{theorem}[Perpendicular Axis Theorem]
    For a two dimensional body on a plane (aka lamina),
    $$I_z=I_x+I_y$$
    where $I_z$ is the moment of inertia along the $z$ axis chosen to be a normal to the plane and $I_x,I_y$ are the moments of inertia along two chosen perpendicular axes on the plane so that all three axes meet at the origin.
\end{theorem}
\begin{proof}
    We have
    $$I_x=\int_A\rho y^2\,\mathrm dA,I_y=\int_A\rho x^2\,\mathrm dA$$
    But
    $$I_z=\int_A\rho r^2\,\mathrm dA=\int_A\rho(x^2+y^2)\,\mathrm dA=I_x+I_y$$
    As desired
\end{proof}
Sometimes the lamina is symmetric enough such that $I_x=I_y$, so $I_z=2I_x$.
This corresponds to the example of a disk.
Note that this theorem works for lamina but does not work for $3$ dimensional bodies.
\begin{theorem}[Parallel Axes Theorem]
    If a rigid body of mass $M$ has moment of inertia $I_c$ about an axis through its center of mass, then for another axis parallel to the original axis with a distance $d$ away, then the moment of inertia $I$ about the new axis is $I=I_c+Md^2$.
\end{theorem}
\begin{proof}
    Choose Cartesian axes such that the centre of mass is at the origin and the rotation axis along $z$-axis.
    Also, choose $x,y$-axes such that the second axes of rotation is through the point $d\underline{\hat{x}}=(d,0,0)$, then
    \begin{align*}
        I_c+Md^2&=\int_V\rho(x^2+y^2)\,\mathrm dV+Md^2\\
        &=\int_V\rho((x-d)^2+y^2)\,\mathrm dV+2d\int_V\rho x\,\mathrm dV\\
        &=\int_V\rho((x-d)^2+y^2)\,\mathrm dV=I
    \end{align*}
    Since the axes are chosen in a way that the origin is the center of mass.
\end{proof}
\begin{example}
    Consider a uniform disk as before with the axis of rotation perpendicular to it through a point on the edge has $I=3Ma^2/2$.
\end{example}
\subsection{Motion of a Rigid Body}
General motion of a rigit body can be described by the composition of translation (of the center of mass) following some trajectory $\underline{R}(t)$ together with a rotation about the center of mass.
Following the previous discussion, we specify points in the body relative to the center of mass by writing $\underline{r_i}=\underline{R}+\underline{s_i}$.
Also recall that $\sum_im_i\underline{r_i}=M\underline{R}$, therefore $\sum_im_i\underline{s_i}=0$.
If a body rotates about its center of mass, with angular velocity $\underline{\omega}$, so $\underline{\dot{s}_i}=\underline{\omega}\times\underline{s_i}$ and $\underline{\dot{r}_i}=\underline{\dot{R}}+\underline{\omega}\times\underline{s_i}$.
The kinetic energy, as we recall, satisfies
$$T=\frac{1}{2}M|\underline{\dot{R}}|^2+\frac{1}{2}\sum_{i=1}^Nm_i|\underline{s_i}|^2=\frac{1}{2}M|\underline{\dot{R}}|^2+\frac{1}{2}I_c\omega^2$$
where $I_c$ is the moment of inertia parallel to $\underline{\omega}$ and through the center of mass.
So $T$ is the sum of translational KE and rotational KE.
We have also shown before that for a general multiparticle system, linear and angular momentum obey $\underline{\dot{P}}=\underline{F},\underline{\dot{L}}=\underline{G}$ where $\underline{F},\underline{G}$ are the total external applied force and torque respectively.
For a rigit body, these two equations determine the translational and rotational motion.
Sometimes, we can exploit the conservation of energy as an easier method of solution.\\
$\underline{L},\underline{G}$ depend on the choice of origin, and we can the origin to be any point fixed in an inertial frame (shown previously).
Or, we can define $\underline{L}$ and $\underline{G}$ about the center of mass, and the equation above, as we shall show, still holds.
Take
\begin{align*}
    \underline{G}&=\frac{\mathrm d}{\mathrm dt}\left(M\underline{R}\times\underline{\dot{R}}+\sum_{i=1}^Nm_i\underline{s_i}\times\underline{\dot{s}_i}\right)\\
    &=M\underline{R}\times\underline{\ddot{R}}+\frac{\mathrm d}{\mathrm dt}\left( \sum_{i=1}^Nm_i\underline{s_i}\times\underline{\dot{s}_i} \right)\\
    &=\underline{R}\times\underline{F}^{\rm ext}+\frac{\mathrm d}{\mathrm dt}\left( \sum_{i=1}^Nm_i\underline{s_i}\times\underline{\dot{s}_i} \right)
\end{align*}
Therefore
\begin{align*}
    \frac{\mathrm d}{\mathrm dt}\left( \sum_{i=1}^Nm_i\underline{s_i}\times\underline{\dot{s}_i} \right)&=\underline{G}-\underline{R}\times\underline{F}^{\rm ext}\\
    &=\sum_{i=1}^N\underline{r_i}\times\underline{F_i}^{\rm ext}-\underline{R}\times\sum_{i=1}^N\underline{F_i}^{\rm ext}\\
    &=\sum_{i=1}^N(\underline{r_i}-\underline{R})\times\underline{F_i}^{\rm ext}\\
    &=\underline{G_c}
\end{align*}
Consider now the motion in a uniform gravitational field with acceleration due to gravity $\underline{g}$, then the total gravitational force and torque acting on a rigit body would be the same as if it is acting on a particle of mass $m$ located in the center of mass (hence it is also called the center of gravity).
So
$$\underline{F}=\sum_{i=1}^N\underline{F_i}^{\rm ext}=\sum_{i=1}^Nm_i\underline{g}=M\underline{g}$$
similarly
$$\underline{G}=\sum_{i=1}^N\underline{G_i}^{\rm ext}=\sum_{i=1}^N\underline{r_i}\times(m_i\underline{g})=M\underline{R}\times\underline{g}$$
Note that the gravitational torque about the center of mass is zero since
$$\underline{G_c}=\sum_{i=1}^N\underline{s_i}\times(m_i\underline{g})=\left( \sum_{i=1}^Nm_i\underline{s_i} \right)\times\underline{g}=0$$
Consider the gravitatioinal potential $-m\underline{r}\cdot\underline{g}$, then
$$V^{\rm ext}=\sum_{i=1}^NV_i^{\rm ext}=\sum_{i=1}^N(-m_i\underline{r_i}\cdot\underline{g})=-M\underline{R}\cdot\underline{G}$$
\begin{example}
    1. Throw a stick in the air.
    So the center of mass follows a parabolic curve and the angular velocity of the stick about center of mass is constant by conservation of energy (or because gravitational torque about center of mass is $0$).\\
    2. A uniform rod of length $l$ and mass $M$ fixed at a pivot point $O$ at one end and makes an angle $\theta$ with the downward vertical.
    We say this is a compound pendulum since the mass is distributed instead of concentrated.
    Consider the angular velocity and angular momentum about the pivot.
    We have $\omega=\dot\theta,L=I\dot\theta=Ml^2\dot\theta/3$.
    So the gravitational torque about $O$ becomes $-Mgl\sin\theta/2$, so $\dot{L}=G\implies I\ddot\theta=-Mgl\sin\theta/2$, so
    $$\ddot\theta=-\frac{3}{2}\frac{g}{l}\sin\theta$$
    which just looks like a simple pendulum of length $2l/3$ (in fact equivalent to it).
    So for small oscillations, the frequency is $f=\sqrt{3g/(2l)}$ and period $2\pi/f$.\\
    Alternatively we can think of the energy, then
    $$E=T+V=\frac{1}{2}I\omega^2-\frac{Mgl}{2}\cos\theta$$
    So
    $$0=\frac{\mathrm dE}{\mathrm dt}=\dot\theta\left(I\ddot\theta+\frac{Mgl}{2}\sin\theta\right)=0$$
    which produces the same result as above.
\end{example}
\subsection{Sliding and Rolling}
Consider a cylinder or sphere with radius $a$ moving along a stationary horizontal surface, then the general motion is a translation of the center of mass with velocity $v$ together with rotation about the center of mass with angular velocity $\omega$.
Let $P$ be the instantaneous point of contact, then the horizontal velocity of this point is given by $v_{\rm slip}=v-a\omega$.\\
There are two extreme cases:\\
1. Pure sliding, where we have $\omega=0,v_{\rm slip}=v\neq 0$.
So the point of contact slips through the surface (probably due to a kinetic frictional force).\\
2. Pure rolling, where we have $\omega,v\neq 0$ but $v_{\rm slip}=v-a\omega=0$.
In this case, the contact point is stationary at any point, which produces rolling without sliding.\\
Instantaneously, we can view the motion of the body as the rotation of the body about the contact point.
Also note that these also apply to inclined plane.
\begin{example}
    Consider a cylinder of radius $a$ and mass $m$ rolling through inclined plane at angle $\alpha$ to the horizontal.
    Let $x$ be the distance down slope travelled by the center of mass, $v=\dot{x}$ and $Mg$ the gravitational force, $N$ the normal reaction and $F$ the frictional force.
    For the cylinder to be purely rolling, we must have $v-a\omega=0$, so $v=a\omega$.\\
    The kinetic energy has
    $$T=\frac{1}{2}Mv^2+\frac{1}{2}I\omega^2=\frac{1}{2}\left(M+\frac{I}{a^2}\right)v^2$$
    Note that due to their directions the normal and frictional force (in the case where $v_{\rm slip}=0$) do no work.
    Now the energy $T+V$ is conserved where $V=-Mgx\sin\alpha$, so
    \begin{align*}
        0&=\frac{\mathrm d(T+V)}{t}\\
        &=\frac{\mathrm d}{\mathrm dt}\left( \frac{M+I/a^2}{2}\dot{x}^2-Mgx\sin\alpha \right)\\
        &=(M+I/a^2)\dot{x}\ddot{x}-Mg\dot{x}\sin\alpha\\
        \implies \left( M+\frac{I}{a^2} \right)\ddot{x}&=Mg\sin\alpha
    \end{align*}
    Note that when $I=0$, this is exactly the equation for a frictionless particle, therefore the rotation makes acceleration smaller.
    Now for a cylinder in question, we have $I=Ma^2/2$, hence
    $$\ddot{x}=\frac{2}{3}g\sin\alpha$$
    We can also obtain the result by using forces and torques.
    By considering the rate of change of linear momentum along the plane, we have $M\dot{v}=Mg\sin\alpha-F$ and the rate of change of angular momentum about the center of mass then gives $I\dot\omega=aF$.
    So as it is rolling, $\dot{v}=a\dot\omega$, whence
    $$M\dot{v}=Mg\sin\alpha-\frac{I\dot{v}}{a^2}$$
    Thus $(M+I/a^2)\dot{v}=Mg\sin\alpha$ as above.\\
    There is yet another way to do this:
    Consider the torque about $P$, we have $I_P=I+Ma^2$ by the parallel axis theroem, also the gravitational torque has $I_P\dot\omega=Mga\sin\alpha$.
    So $v=a\omega$ gives $(I+Ma^2)\dot{v}/a=Mga\sin\alpha$.
\end{example}
\begin{example}
    We want to study the transition from a sliding motion to a rolling one.
    Consider a snooker ball on a horizontal plane hit by a cue instantaneously which gives it an initial velocity $v_0$.
    Initially $v=v_0$ and $\omega_0$, where sliding occurs (so no rotation at $t=0$).
    The kinetic frictional force obeys $F=\mu N=\mu Mg$ where $\mu$ is a constant (coefficient of kinetic friction).
    The linear motion has $M\dot{v}=-F$ and the angular motion $I\dot\omega=aF$.
    Also for a sphere $I=2Ma^2/5$, hence we have, by integrating,
    $$\begin{cases}
        v=v_0-\mu gt\\
        \omega=5\mu gt/(2a)
    \end{cases}$$
    So when the ball is still moving,
    $$0\le v_{\rm slip}=v-a\omega=v_0-\frac{7}{2}\mu gt$$
    So the total time of rolling is $t_{\rm roll}=2v_0/(7\mu g)$.
    During $0\le t\le t_{\rm roll}$, the friction acts to decrease $v$ and increase $\omega$ till the no-slip condition is satisfied, when $t=t_{\rm roll}$ and $v=v_{\rm roll}=5v_0/7$.
    But at $t_{\rm roll}$, the rolling could as well persist but the friction does no further work.
    At $t=t_{\rm roll}$, the kinetic energy is
    $$T=\frac{1}{2}Mv^2+\frac{1}{2}I\omega^2=\frac{1}{2}M\left( 1+\frac{2}{5} \right)v_{\rm roll}^2=\frac{5}{7}\left( \frac{1}{2}Mv_0^2 \right)$$
    So the loss of KE due to friction has a total of
    $$\int_0^{t_{\rm roll}}Fv_{\rm slip}\,\mathrm dt=\int_0^{t_{\rm roll}}F\left( v_0-\frac{7}{2}\mu gt \right)\,\mathrm dt=\frac{1}{7}Mv_0^2$$
\end{example}
