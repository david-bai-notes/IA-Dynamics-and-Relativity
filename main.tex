\documentclass[a4paper]{article}

\usepackage{hyperref}

\newcommand{\triposcourse}{Dynamics and Relativity}
\newcommand{\triposterm}{Lent 2020}
\newcommand{\triposlecturer}{Prof. P. H. Haynes}
\newcommand{\tripospart}{IA}

\usepackage{amsmath}
\usepackage{amssymb}
\usepackage{amsthm}
\usepackage{mathrsfs}

\theoremstyle{plain}
\newtheorem{theorem}{Theorem}[section]
\newtheorem{lemma}[theorem]{Lemma}
\newtheorem{proposition}[theorem]{Proposition}
\newtheorem{corollary}[theorem]{Corollary}
\newtheorem{problem}[theorem]{Problem}
\newtheorem*{claim}{Claim}

\theoremstyle{definition}
\newtheorem{definition}{Definition}[section]
\newtheorem{conjecture}{Conjecture}[section]
\newtheorem{example}{Example}[section]
\newtheorem*{law}{Law}
\newtheorem*{postulate}{Postulate}

\theoremstyle{remark}
\newtheorem*{remark}{Remark}
\newtheorem*{note}{Note}

\title{\triposcourse{}
\thanks{Based on the lectures under the same name taught by \triposlecturer{} in \triposterm{}.}}
\author{Zhiyuan Bai}
\date{Compiled on \today}

%\setcounter{section}{-1}

\begin{document}
    \maketitle
    This document serves as a set of revision materials for the Cambridge Mathematical Tripos Part \tripospart{} course \textit{\triposcourse{}} in \triposterm{}.
    However, despite its primary focus, readers should note that it is NOT a verbatim recall of the lectures, since the author might have made further amendments in the content.
    Therefore, there should always be provisions for errors and typos while this material is being used.
    \tableofcontents
    \section{Newtonian Dynamics: The Basics}
\subsection{Particles}
\begin{definition}
    A particle is an object that has negligible size but have positive mass $m$ and electric charge $q$.
\end{definition}
Since a particle will have small size, we can describe its position by a simple position vector $\underline{r}(t)\in\mathbb R^3$ relative to the origin.
We often write the vector in terms of its Cartesian components $\underline{r}=x\underline{i}+y\underline{j}+z\underline{k}=(x,y,z)$ where $\underline{i},\underline{j},\underline{k}$ are an orthonormal basis.
The choice of the coordinate system (the origin and the basis) defines a frame of reference.\\
When the particle moves, its position is determined by a curve $\underline{r}(t)$.
The velocity of the particle is naturally its derivative $\underline{u}(t)=\underline{\dot{r}}(t)$.
Geometrically, the velocity will be the tangent to the curve (or trajectory) at time $t$.
The momentum as we know would be $\underline{p}=m\underline{u}=m\underline{\dot{r}}$.
The acceleration is defined as $\underline{a}=\underline{\dot{u}}=\underline{\ddot{r}}$.
\begin{note}
    The time derivative of a vector valued function $\underline{v}(t)$ is
    $$\underline{\dot{v}}(t)=\lim_{h\to0}\frac{\underline{v}(t+h)-\underline{v}(t)}{h}$$
    provided its existence.
    If anyone is worried, $\underline{v}\to\underline{v_0}\iff\|\underline{v}-\underline{v_0}\|\to 0$.\\
    In particular, if $\underline{v}=x\underline{i}+y\underline{j}+z\underline{k}$, then $\underline{\dot{v}}=\dot{x}\underline{i}+\dot{y}\underline{j}+\dot{z}\underline{k}$ (given that the frame of reference is invariance in time).
\end{note}
\begin{proposition}
    For scalar functions $f(t)$ and vector functions $\underline{g}(t),\underline{h}(t)$, we have\\
    1. $(f\underline{g})^\prime=f^\prime \underline{g}+f\underline{g}^\prime$.\\
    2. $(\underline{g}\cdot\underline{h})^\prime=\underline{g}^\prime\cdot\underline{h}+\underline{g}\cdot\underline{h}^\prime$.\\
    3. $(\underline{g}\times\underline{h})^\prime=\underline{g}^\prime\times\underline{h}+\underline{g}\times\underline{h}^\prime$.\\
    Note that sometimes the order matters.
\end{proposition}
\subsection{Newton's Laws of Motion}
\begin{law}[Newton's First Law]
    There exists inertial frames of reference (or inertial frames).
    That is, a particle at rest or move in constant velocity continues to do so given that it is acted by no force.
\end{law}
\begin{law}[Newton's Second Law]
    In an inertial frame, then the motion obeys the rule $\underline{\dot{p}}=\underline{F}$.
\end{law}
\begin{law}[Newton's Third Law]
    To every action there is an equal and opposite reaction.
\end{law}
The statements, albeit are made for particles, can be extended to finite bodies.
\footnote{Bounded bodies.}
\subsection{Inertial Frames and Galileo Transformation}
If we have an inertial frame, $\ddot{r}=0$ if there is no force acting on it.
There is obviously not only one inertial frame.
In particular, if $S$ is an inertial frame, then a frame $S'$ moving with uniform velocity relative to $S$ is also an inertial frame.
For example, if the frame $S'$ is moving with velocity $v$ on the $x$ direction, then
$$\begin{cases}
    x'=x-vt\\
    y'=y\\
    z'=z\\
    t'=t
\end{cases}$$
More generally, if $S'$ is moving with vector velocity $\underline{v}$ relative to $S$, we have
$$\begin{cases}
    \underline{r'}=\underline{r}-\underline{v}t\\
    t'=t
\end{cases}$$
This transformation is called a \textit{boost}.
For a partical having position vector $\underline{r}(t)$ in $S$ and $\underline{r'}(t')$ in $S'$.
So we have the velocity $\underline{u'}=\underline{u}-\underline{v}$ (note that the primes are NOT used for derivatives here) and $\underline{a'}=\underline{a}$.
\begin{definition}
    A general Galileo transformation is one which preserves inertial frames.
    It combines a boost with any of the following:\\
    1. Translation of space: $\underline{r'}=\underline{r}-\underline{r_0}$.\\
    2. Translation of time: $t'=t-t_0$.\\
    3. Rotations and reflections: $\underline{r'}=R\underline{r},R\in\operatorname{O}(3)$.\\
    This set generates the Galilean group of transformations.
\end{definition}
Note that if the acceleration is zero in one frame, so it is in another.
\begin{definition}[Principle of Galilean Relativity]
    The laws of (Newtonian) physics is unchanged in all inertial frames.
\end{definition}
That is, the laws of physics look the same in every inertial frame.
Hence the system of Newtonian physics has to be invariant under the Galilean transformations.
\subsection{Newton's Second Law}
The law postulates that $\underline{F}=\underline{\dot{p}}$.
Assume that $m$ is constant in time, then we have $\underline{F}=m\underline{\ddot{r}}$.
Easily $m$ is the measure of ``reluctance to accelerate'', that is inertia.
If we specify $\underline{F}$ as a function of $\underline{r},\underline{\dot{r}},t$, then we have a second order ODE in $\underline{r}$:
$$\underline{F}(\underline{r},\underline{\dot{r}},t)=m\underline{\ddot{r}}$$
We then need two initial conditions to solve the equation (or to determine the motion).
For example, we can specify the initial position and velocity.
With these information
\footnote{And perhaps Picard-Lindel\"of Theorem}
we can get an unique solution for the trajectory of our particle.
\subsection{Examples of Forces}
Consider $2$ particles indexed by $1,2$, then Newton tells us
\begin{law}[Newton's Law of Gravitation]
    There is an action-reaction pair on the two particles, namely
    $$\underline{F_1}=-\frac{Gm_1m_2}{|\underline{r_1}-\underline{r_2}|^3}(\underline{r_1}-\underline{r_2})=-F_2$$
\end{law}
In particular, $|\underline{F_1}|=|\underline{F_2}|\propto|\underline{r_1}-\underline{r_2}|^{-2}$.
This is known as the inverse square law.
It is quite obvious that $G$ has an unit.
It is called Newton's Gravitation Constant.\\
Another example is electromagnetic forces.
Let there be a particle with electric charge $q$ and imagine that it is moving in an electric-magnetic field $\underline{E}(\underline{r},t)$ and $\underline{B}(\underline{r},t)$.
\begin{law}[Lorentz Force Law]
    We have
    $$\underline{F}=q(\underline{E}+\underline{\dot{r}}\times\underline{B})$$
\end{law}
\begin{example}
    Take $\underline{E}=\underline{0},\underline{B}=\underline{B}(t)$, i.e. the electric field is constant and the magnetic field is constant in space.
    Hence
    $$m\underline{\ddot{r}}=q\underline{\dot{r}}\times\underline{B}(t)$$
    Choose axes such that $\underline{B}=B\underline{\hat{z}}$, then $m\ddot{z}=0\implies z=z_0+ut$.
    As for the other directions, we have
    $$\begin{cases}
        m\ddot{x}=qB\dot{y}\\
        m\ddot{y}=-qB\dot{x}
    \end{cases}$$
    which we can easily solve to get
    $$\begin{cases}
        x=x_0-\alpha\cos(\omega(t-t_0))\\
        y=y_0+\alpha\sin(\omega(t-t_0))
    \end{cases}$$
    which shall produce a helical path which is clockwise when viewed from the direction of $\underline{B}$.
    And the axis of the helix is parallel to the magnetic field.
\end{example}
    \section{Dimensional Analysis}
\subsection{Basic Dimensional Quantities and Units}
For most motions we will be considering, there are basically three dimensions of interests: length ($L$), mass ($M$) and time ($T$).
In general, the dimension of a physical quantity $X$ can be expressed using thsese three dimensions.
For example, the density can be expressed by $ML^{-3}$, and the force can be expressed by $MLT^{-2}$.
We are only going to consider the product of powers of the dimensional quantities.
We can then introduce units for the basic dimensional quantities.
Most likely we will use the SI unit system ($L=$m, $M=$kg, $T=$s).
For other quantities, we can form units out of the basic units we defined for the basic quantities.
\begin{example}
    To find the unit of the constant $G$ in Newton's Law of Gravitation, we can determine by writing each quantity in basic quantities, so we have $G=L^3T^{-2}M^{-1}$, therefore the unit for $G$ will be ${\rm m^3s^{-2}kg^{-1}}$.
\end{example}
The general principle is that dynamical or physical equations must work for any chosen system of units.
\subsection{Scaling}
Suppose I have a dimensional quantity $Y$ which depends on some other quantities $X_1,X_2,\ldots,X_n$.
Let the diensions of the quantity $Y$ be $L^aM^bT^c$, and $X_i$ has dimensions $L^{a_i}M^{b_i}T^{c_i}$.
We want to determine the dimensions of $Y$ from that of $X_i$.
So obviously we have $Y=C\prod_i X_i^{p_i}$, so
$$\begin{cases}
    a=\sum_ip_ia_i\\
    b=\sum_ip_ib_i\\
    c=\sum_ip_ic_i
\end{cases}$$
If $n=3$, then there is an unique solution iff $X_1,X_2,X_3$ are independent, so
$$\begin{vmatrix}
    a_1&a_2&a_3\\
    b_1&b_2&b_3\\
    c_1&c_2&c_3
\end{vmatrix}\neq 0$$
which happens most of the time.
Note that for general $n$ there must be a solution (not necessarily unique) if we assume that $Y$ does indeed depend on a subset of $\{X_i\}$.
So for $n<3$, there is an unique solution as well.\\
For $n>3$, however, we can choose $n-3$ dimensionless constants
$$\lambda_i=X_i/(X_1^{p_{i1}}X_2^{p_{i2}}X_3^{p_{i3}})$$
where $i=4,5,\ldots$, assuming $X_1,X_2,X_3$ are independent.
So
$$Y=C(\lambda_4.\lambda_5,\ldots)X_1^{p_1}X_2^{p_2}X_3^{p_3}$$
where $C$ is a dimensionless function.
This is sometimes known as Bridgemzn's Theorem.
\begin{example}
    Consider a simple pendulum.
    Let $d$ be the horizontal initial displacement, $m$ the mass, and $g$ the acceleration due to gravity, and $l$ the length of the string, and we want to find expression of the period $P$ in term of these.
    Speaking of dimensions,
    $$\begin{cases}
        [P]=T\\
        [d,l]=L\\
        [g]=LT^{-2}\\
        [m]=M
    \end{cases}$$
    Thus
    $$T=M^{p_1}L^{p_2}(LT^{-2})^{p_3}$$
    solve to get $p_1=0,p_2=1/2,p_3=-1/2$
    Hence
    $$P=C\left( \frac{d}{l} \right)\sqrt{\frac{l}{g}}$$
    For a dimensionless function $C$.\\
    Hence, if we scale $d,l$ by $2$, the period will be scaled by $\sqrt{2}$.
    Also $P$ is independent of $m$.
\end{example}
\begin{example}
    Taylor's estimate to the first atomic explosion.\\
    We want to estimate the radius of the fireball $R$ which has dimension $L$.
    $R$ depends on the time $t$ since the explosion which has dimension $T$.
    The density of air $\rho_0$ which has dimension $ML^{-3}$ is also involved.
    Lastly the energy of explosion $E$ having dimension $ML^2T^{-2}$.\\
    So by doing dimensional analysis, we immediately (since there are only $3$ depending dimensions) have $R\propto \sqrt[5]{Et^2/\rho_0}$.
    This has allowed Taylor to estimate the size of $E$.
\end{example}
    \section{Forces}
\subsection{Force and Potential Energy in One (Spacial) Dimension}
Consider a point mass $m$ moving on a straight line with position given by $x(t)$.
We assume that the force $F=F(x)$ depends entirely on position, not velocity and time.
\begin{definition}
    The potential energy $V(x)$ is any function that satisfies $F(x)=-\mathrm dV/dx$.
\end{definition}
Equivalently,
$$V(x)=V(0)+\int_0^xF(x)\,\mathrm dx$$
where $V(0)$ can be taken arbitrarily.
The equation of motion is simply $m\ddot{x}=-\mathrm dV/dx$ by Newton's Second Law.
\begin{definition}
    The KInetic energy $T$ is defined by $T=m|\dot{x}|^2/2$
\end{definition}
\begin{theorem}
    Under the assumptions and definitions above, we have $\mathrm d(T+V)/\mathrm dt=0$.
\end{theorem}
\begin{proof}
    $$\frac{\mathrm d(T+V)}{\mathrm dt}=\frac{2m\dot{x}\ddot{x}}{2}+\frac{\mathrm dV}{\mathrm dx}\frac{\mathrm dx}{\mathrm dt}=\dot{x}(m\ddot{x}+\frac{\mathrm dV}{\mathrm dx})=0$$
    By Newton's Second Law.
\end{proof}
Note that if we lost that restriction on the time and velocity independence of the force, we lose the conservation of energy in general.
\begin{example}
    Consider a harmonic oscillator, so $F(x)=-kx$ where $k$ is a positive constant.
    So $V(x)=kx^2/2$ by choosing the arbitrary constant as $0$.
    We want to calculate all the stuff to verify the conservation of energy.
    \footnote{Hey, you literally just proved it.}
    We can solve the motion by solving $m\ddot{x}=-kx$ which solves to $x=A\sin(\sqrt{k/m}t)+B\cos(\sqrt{k/m}t)$.
    And plugging in gives $\mathrm dE/\mathrm dt=0$.
\end{example}
As the first instance of Newton's Second Law, the conservation of energy is a useful rule to determine a one dimensional motion.
Using conservation of energy, we have
$$\dot{x}=\pm\sqrt{\frac{2}{m}(E-V(x))}$$
which is a first order ODE.
So
$$\int_{x_0}^x\frac{\mathrm du}{\sqrt{2(E-V(u))/m}}=t-t_0$$
where $x(t_0)=x_0$.
In principle we can solve it to obtain the motion.\\
We can also have some qualitative insight from conservation of energy.
Consider $V(x)=\lambda(x^3-3\beta^2x)$ where $\lambda,\beta>0$ are constants.
We can sketch the potential energy to find that $V$ has a local maximum at $-\beta$, which value happens again at $x=2\beta$.
And it has a local minimum at $\beta$, where it again obtain the same value at $-2\beta$.
So we can find certain properties of the motion from the graph if the motion start at rest, then we must have $V(x)\le V(x_0)$\\
Case 1: $x_0<-\beta$, it will moves to left so as to reduce the potential and gain speed.\\
Case 2: $-\beta<x_0<2\beta$, then the particle will be restricted in the region $-\beta<x_0<2\beta$ and will oscillate.\\
Case 3: $x_0>2\beta$, then it will move to the right.\\
The case becomes special if we turn to the stationary (or equilibrium) points.
Obviously $x_0=-\beta$ is an unstable fixed point and $x_0=\beta$ is a stable fixed points.
So at $x_0=2\beta$, it will end its notion at the fixed point $x=-\beta$.
In this case, we can analyse the behaviour by writing down the integral.
This can show that the time to reach $x=-\beta$ is infinite when we approach to $2\beta$.
\subsection{Equilibriums}
The points $x=\pm\beta$ in this case are called equilibrium points, at which the particle can always stay at rest.
The condition for this to happen is $V^\prime(x_0)=0$.
We are going to analyze the motion near the equilibrium at $x_0$ by expanding its Taylor series
$$V(x)\approx V(x_0)+(x-x_0)V^\prime(x_0)+\frac{(x-x_0)^2}{2}V^{\prime\prime}(x_0)=V(x_0)+\frac{(x-x_0)^2}{2}V^{\prime\prime}(x_0)$$
We assume for a moment that $V^{\prime\prime}(x_0)$ does not vanish.
\footnote{If it does vanish, we will have to look at higher order terms.}
So we have $m\ddot{x}=-(x-x_0)V^{\prime\prime}(x_0)$.\\
If $V^{\prime\prime}(x_0)>0$, it's a minimum of $V$ which produces the equation of a harmonic oscillator with period $\sqrt{V^{\prime\prime}(x_0)/m}$.
In this case, we say it is a stable equilibrium.\\
If $V^{\prime\prime}(x_0)<0$, it's a maximum of $V$ which produces the equation of an exponentially growing solution.
Hence it is an unstable equilibrium with growth rate $\sqrt{-V^{\prime\prime}(x_0)/m}$.
\begin{example}
    We look back to a pendulum with mass $m$, length $l$ and angle $\theta$.
    If we think of Newton's Second Law, one can obtain
    $$F=ml\ddot{\theta}=-mg\sin\theta=-\frac{\mathrm d}{\mathrm d\theta}(-mg\cos\theta)$$
    So we have $E=T+V=ml^2\dot{\theta}^2/2-mgl\cos\theta$.
    One can check that $\dot{E}=0$.
    Also the potential $V(\theta)=-mg\cos\theta$ has stable equilibrium at $\theta=2\pi k,k\in\mathbb Z$ and unstable at $\theta=\pi+2\pi k,k\in\mathbb Z$.
    So if the initial value of $\theta$ is in $(-\pi,\pi)$ (or $|V|<mgl$), the pendulum will oscillate.
    If $|V|>mgl$, then it will go round and round.\\
    Now we want to analyze the period of oscillations.
    Suppose the original angle is at $\theta_0\in (0,\pi)$, then the oscillation is going to be $\theta_0\to 0\to-\theta_0\to0\to\theta_0$, so the period is $4$ times the time taken for $\theta_0$ to $0$, hence
    $$P=4\int_0^{\theta_0}\frac{\mathrm d\theta}{\sqrt{2gl(\cos\theta-\cos\theta_0)/l^2}}=4\sqrt{\frac{l}{g}}\int_0^{\theta_0}\frac{\mathrm d\theta}{\sqrt{2\cos\theta-2\cos\theta_0}}=\sqrt{\frac{l}{g}}F(\theta_0)$$
    For small $\theta_0$ we have
    $$F(\theta_0)\approx 4\int_0^{\theta_0}\frac{\mathrm d\theta}{\sqrt{\theta_0^2-\theta^2}}=2\pi$$
    Hence $P\approx 2\pi\sqrt{l/g}$.
\end{example}
\subsection{Force and Potential in Three Dimensions}
Consider a particle $\underline{r}$ in motion in three dimensional space.
Then $m\underline{\ddot{r}}$ and $T=m|\underline{\dot{r}}|^2/2$.
And the rate of change of $T$ is then
$$\frac{\mathrm dT}{\mathrm dt}=m\underline{\dot{r}}\cdot\underline{\ddot{r}}=\underline{\dot{r}}\cdot\underline{F}$$
Suppose the particle tranverse a path $C$ from $t_0$ to $t_1$, then
\begin{definition}
    The work done is
    $$\int_{t_0}^{t_1}\underline{F}\cdot\underline{\dot{r}}\,\mathrm dt=\int_C\underline{F}\cdot\mathrm d\underline{r}$$
\end{definition}
We can also write that the total work equals
$$\int_C\underline{F}\cdot\mathrm d\underline{r}=\int_C F_x\,\mathrm dx+F_y\,\mathrm dy+F_z\,\mathrm dz$$
Suppose that the force is a function of the position $\underline{F}(\underline{r})$ (also called a force field).
\begin{definition}
    A force field $\underline{F}(\underline{r})$ is called conservative if $\underline{F}=\nabla V$ for some $V:\mathbb R^3\to\mathbb R$.
\end{definition}
If a force field $\underline{F}(\underline{r})$ is conservative, we say $V$ is the potential function.
Also in this case $E=T+V(\underline{r})$ conserved.
Indeed,
$$\frac{\mathrm dE}{\mathrm dt}=\frac{\mathrm dT}{\mathrm dt}+\frac{\mathrm dV}{\mathrm dt}=m\underline{\dot{r}}\cdot\underline{\ddot{r}}+\nabla V\cdot\underline{\dot{r}}=\underline{\dot{r}}\cdot(m\underline{\ddot{r}}-\underline{F})=0$$
The total work done by a conservative force $\underline{F}$ is
$$\int_C\underline{F}\cdot\mathrm d\underline{r}=\int_C-\nabla V\cdot\mathrm d\underline{r}=V(\underline{r}(t_0))-V(\underline{r}(t_1))$$
So the work done is independent of the path taken.\\
In particular, if the curve is closed, no work is done.\\
$\underline{F}$ is conservative if $\nabla\times\underline{F}=\underline{0}$ (given that the domain is simply connected).
\subsection{Angular Momentum}
\begin{definition}
    The angular momentum for a particle with mass $m$ and velocity $\underline{\dot{r}}$ is defined as
    $$\underline{L}=\underline{r}\times\underline{p}=m\underline{r}\times\underline{\dot{r}}$$
    And
    $$\underline{G}=\frac{\mathrm d\underline{L}}{\mathrm dt}=m\underline{\dot{r}}\times\underline{\dot{r}}+m\underline{r}\times\underline{\ddot{r}}=\underline{r}\times\underline{F}$$
    is defined as the torque, or moment of force.
\end{definition}
Note that $\underline{L},\underline{G}$ both depend on the choice of origin, so we must specify them when talking about angular stuff.
\begin{remark}
    If $\underline{r}\times\underline{F}=\underline{0}$ then $\underline{G}=0$, thus $\underline{L}$ is constant.
    In this case, we say the angular momentum is conserved.
\end{remark}
\subsection{Central Forces}
A special type of conservative force occurs when the potential $V$ depends entirely on $|\underline{r}|$, so $V(\underline{r})=V(|\underline{r}|)=V(r)$,
\footnote{It's just a tiny abuse of notation. vErY tInY.}
$$\underline{F}(\underline{r})=-\nabla V(|\underline{r}|)=-\frac{\mathrm dV}{\mathrm dr}\underline{\hat{r}},\underline{\hat{r}}=\frac{\underline{r}}{|\underline{r}|}$$
So $\underline{F}$ and $\underline{r}$ are parallel, therefore $\mathrm d\underline{L}/\mathrm dt=\underline{G}=\underline{F}\times\underline{r}=0$.
\subsection{Gravity}
Recall that Newton's Gravitational Law states
$$V=-\frac{GMm}{|\underline{r}|},\underline{F}=-\nabla V=-\frac{GMm}{|\underline{r}|^2}\underline{\hat{r}}=-\frac{GMm}{|\underline{r}|^3}\underline{r}$$
Note that the $m$ here can be ignored (in the way shown below) if we are only interested in the motion due to Newton's Second law:
\begin{definition}
    The gravitational potential is defined by $\Phi_g(\underline{r})=V/M=-GM/|\underline{r}|$, and the gravitational field by $\underline{g}=-\nabla\Phi_g(\underline{r})=-GM\underline{\hat{r}}/r^2$.
\end{definition}
The gravitational field and its potential, as functions, are dependent of $M$ alone.
We also have $m\Phi_g=V,m\underline{g}=\underline{F}$.
For a set of more than one masses, we can simple generalize by adding the corresponding fields and potential together by the superposition principle.
Hence for continuous bodies, we can replace the sum by an integral.
In particular, if the body is spherical with radius $R$ and we have $|\underline{r}|>R$, then we do have $\Phi_g(\underline{r})=-GM/|\underline{r}|$, thus spherical bodies do behave like a point when measuring from above its surface.
\begin{note}
    The mass $m$ in Newton's Second Law $m\underline{\ddot{r}}=\underline{F}$ is called the inertial mass, whilist the mass in Newton's Gravitational Law $\underline{F}=-GMm\underline{r}/|\underline{r}|^2$ is the gravitational mass.
    These two definitions of mass are different in relativity but are very closely related (about a difference of $10^{-12}$).
    The precise difference will be discussed in General Relativity.
\end{note}
There are a few results about the effect of gravity.
\begin{example}[Potential Energy near the Surface]
    For a mass $m$ at height $z$ above a spherical mass $M$ with radius $R$, if $z<<R$, then the potential energy is given by
    \begin{align*}
        V(R+z)&=-\frac{GMm}{R+z}\\
        &=-\frac{GMm}{R}+\frac{GMm}{R^2}z+o(R^{-2})\\
        &\approx -\frac{GMm}{R}+mgz\\
        &=\text{const}+mgz
    \end{align*}
    For earth, we have the approximation $g\approx 9.8{\rm ms^{-2}}$
\end{example}
\begin{example}[Escape Velocity]
    We want to find the critical velocity $\underline{v}$, perpendicular to $\underline{r}$, to leave a planet.
    Due to the conservation of energy $E=T+V=m|\underline{v}|^2/2-GMm/R$, the particle can escape (i.e. $\underline{v}$ is nonnegative at infinity) iff the initial energy has $E_0\ge 0$, which happens iff
    $$m|\underline{v}|^2/2\ge GMm/R\implies |\underline{v}|\ge \sqrt{\frac{2GM}{R}}=v_{\rm esc}$$
\end{example}
\subsection{Electromagnetic Forces}
We have seen previously that for a point charge $q$, the force has the expression $\underline{F}=q(\underline{E}+\underline{\dot{r}}\times\underline{B})$
In general $\underline{E},\underline{B}$ are functions of $\underline{r}$ and $\underline{t}$.
These are known as the Lorentz Force Law.
For convenience or something, we are going to restrict ourselves to time-independent fields.
So we want the electric field to be conservative, i.e. $\underline{E}=-\nabla\Phi_e$ where $\Phi_e$ is called the electrostatic potential.
\begin{claim}
    In a time-independent electromagnetic field, the energy
    $$E=T+V=\frac{m|\underline{\dot{r}}|^2}{2}+q\Phi_e(\underline{r})$$
    is conserved.
\end{claim}
\begin{proof}
    \begin{align*}
        \frac{\mathrm dE}{\mathrm dt}&=\frac{\mathrm d}{\mathrm dt}\left(\frac{m|\underline{\dot{r}}|^2}{2}+q\Phi_e(\underline{r})\right)\\
        &=m\underline{\dot{r}}\cdot\underline{\ddot{r}}-q\underline{\dot{r}}\cdot\underline{E}\\
        &=\underline{\dot{r}}(m\underline{\ddot{r}}-q\underline{E})\\
        &=0
    \end{align*}
    So $E$ is constant.
\end{proof}
\begin{law}
    Now consider a point charge $Q$ located at the origin.
    It generates an electrostatic field
    $$\Phi_e(\underline{r})=\frac{Q}{4\pi\epsilon_0|\underline{r}|},\underline{E}=\frac{Q}{4\pi\epsilon_0|\underline{r}|^2}\underline{\hat{r}}$$
    where $\epsilon_0$ is called the electric constant.
\end{law}
So the force exerted on our point charge $q$ is
$$\underline{F}=q\underline{E}=\frac{Qq}{4\pi\epsilon_0|\underline{r}|^2}\underline{\hat{r}}$$
which is called the Coulomb force.
One observe the similarity of this with the gravitational law (inverse-square law).
Also, by considering the signs, we find that same signed charges repel, opposite charges attract.
\subsection{Friction}
The friction is a contact force, which occurs when two body touches each other (they may not be of the same form though).
It is a convenient description of complicated molecular-scale physics.
So friction is not a kind of fundamental forces (gravity, EM, strong force, weak force).\\
We first consider a special kind of friction that is \textit{dry friction}.
Solids stand on each other exerts an action and reaction pair of normal forces (normal to the surface of contact), which prevents an object from merging with the other.
There is also a tangential force, which is a tangent to the trajectory on the surface where the solid is moving.\\
Now imagine we place a block on a slope.
If it remains at rest, the tangential force is called the static force, which exists even without relative motion.
In this case we have
\begin{law}
    The static force $\underline{F}$ has $|\underline{F}|\le\mu_s|\underline{N}|$ where $\mu_s$ is a constant (depending on the materials) called the coefficient of static friction, and $\underline{N}$ is the normal force.
\end{law}
So the block can rest on the plane provided that $\alpha\le\tan^{-1}(\mu_s)$ where $\alpha$ is the inclination.\\
There is a kinetic frictional force as well, which depends on the kinetic motion of the object.
\begin{law}
    The kinetic frictional force has $\underline{F}=\mu_k|\underline{N}|$ where $\mu_k$ is also a constant depending on the materials.
\end{law}
Normally $\mu_s>\mu_k>0$.\\
The most complicated type of friction is the \textit{fluid drag}, which is the friction exerted by a solid moving in fluid medium.
The model of linear drag says that
$$\underline{F}=-k_1\underline{\underline{u}}$$
where $\underline{u}$ is the velocity along the direction of motion and $k$ is a constant.
This model is relevent if we are considering a small object moving through a viscous fluid.
\begin{law}[Stokes' Law]
    Consider a sphere with radius $R$ moving in a viscous fluid with viscosity of $\eta$, then we have $k_1=6\pi\eta R$.
\end{law}
Another drag regime is called the quadratic model, which is for large bodies moving in less viscous fluid.
$$\underline{F}=-k_2|\underline{u}|\underline{u}$$
Typically we have $\rho R^2C_0$ where $\rho$ is the density of the fluid, $R$ is the radius and $C_0$ is the drag coefficient.\\
In the case of a linear drag, the rate of work done is $\underline{F}\cdot\underline{u}=-k_1|\underline{u}|^2$ and for quadratic law $\underline{F}\cdot\underline{u}=-k_2|\underline{u}|^3$.
The fluid gains energy due to this energy loss by the solid, obviously.\\
Recall from differential equations that the damp oscillator $m\ddot{x}=-kx-\lambda\dot{x}$ where the last term is a drag.
We know how to solve this.
\begin{example}
    Projectiles moving under uniform gravity and experiencing linear drag force.
    The equation of motion is hence
    $$m\ddot{x}=m\underline{g}-k\underline{\dot{x}}$$
    Consider the particle start at the origin with some velocity $\underline{U}$, so the initial conditions are $\underline{x}(0)=\underline{0},\underline{\dot{x}}(0)=\underline{U}$.
    We can solve the equation in $\underline{\dot{x}}$ and substituting the initial condition gives
    $$\underline{\dot{x}}=\frac{m\underline{g}}{k}+(\underline{U}-m\underline{g}/k)e^{-kt/m}$$
    Integrate it again and plug in the other initial condition,
    $$\underline{x}=\frac{m\underline{g}}{k}t+\frac{m}{k}(\underline{U}-m\underline{g}/k)(1-e^{-kt/m})$$
    Set $\underline{x}=(x,y,z),\underline{U}=U(\cos\theta,0,\sin\theta),\underline{g}=(0,0,-g)$.
    Hence we have, by simply plugging things in, that $y$ direction is irrelevant at all, while
    $$\dot{x}=U\cos\theta e^{-kt/m},\dot{y}=0,\dot{z}=(U\sin\theta+mg/k)e^{-kt/m}-mg/k$$
    so the $x$-velocity will eventually go to $0$ and $z$ velocity to a terminal value $mg/k$ (roughly after $t=m/k$).\\
    As for displacement, we have
    $$x=mU\cos\theta/k(1-e^{-kt/m}),z=-mgt/k+m/k(U\sin\theta+mg/k)(1-e^{-kt/m})$$
    so $x$ is bounded but $z$ is eventually moving (linearly as $t$ large).\\
    Now we want to turn to analyze the range $R(U,\theta,m,k,g)$ of the projectile for it to reach its original position (assuming we project it upwards), so by dimensional analysis we get the dimensionless quantity to be $f(\theta,kU/mg)=f(\theta,(U/g)/(m/k))$.
    Note that $U/g$ is proportional to the time taken to reduce velocity such that it vanishes; and $m/k$ is the approximate time to achieve the terminal velocity.
    So weak friction means $kU/mg<<1$ and strong firction means $kU/mg>>1$.\\
    We have $R=U^2/gf(\theta,kU/mg)$.
    If $kU/mg<<1$, then $R\approx U^2/g(2\sin\theta\cos\theta)$.
    If $kU/mg>>1$, then $R\approx U^2/g(\cos\theta(mg/kU))$.
\end{example}
    \section{Orbits}
The study of orbits it is motivated by the motion of heavenly bodies under the infludence of the gravitational force due to e.g. a star.
Of course, we want to study a conservative field $-\nabla V$ where $V$ is a potential of a central force (so it only depends on $r=|\underline{r}|$), that is
$$m\underline{\ddot{r}}=-\nabla V(r)$$
We shall study the case when the central body is much more massive than the orbiting body, so that the central body can be regarded as fixed.\\
Recall that $\underline{L}=m\underline{r}\times\underline{\dot{r}}$, and in the case for a central force $\underline{\dot{L}}=0$, so $\underline{L}$ is constant.
Also we always have $\underline{L}\cdot\underline{r}=0$, so we can regard the motion as if it is in a plane.
\subsection{Polar Coordinates in a Plane}
In an orbit problem, we of course want to use polar coordinate to simplify calculation.
Since we can regard the problem as two-dimensional, we can use the plane polar coordinates $x=r\cos\theta,y=r\sin\theta$, so we define the unit vectors
$$\underline{e_r}=(\cos\theta,\sin\theta)^\top,\underline{e_\theta}=(-\sin\theta,\cos\theta)^\top$$
So we can use $\underline{e_r},\underline{e_\theta}$ as two basis vectors, but note that they are dependent of the position.
Note that $\underline{e_r}$ is always in the direction of the position and $\underline{e_\theta}$ the direction of rotation.
Also note that $\mathrm d\underline{e_r}/\mathrm d\theta=\underline{e_\theta},\mathrm d\underline{e_\theta}/\mathrm d\theta=-\underline{e_r}$, so
$$\frac{\mathrm d\underline{e_r}}{\mathrm dt}=\underline{e_\theta}\dot{\theta},\frac{\mathrm d\underline{e_\theta}}{\mathrm dt}=-\underline{e_r}\dot{\theta}$$
Now we turn to consider the implications for the velocity of the particle given some acceleration.
Write $\underline{r}=r\underline{e_r}$.
Consider this as a function of time, then $\underline{v}=\underline{\dot{r}}=\dot{r}\underline{e_r}+r\underline{e_\theta}\dot{\theta}$.
$\dot{r}$ is the radial component of the velocity while $\dot{\theta}$ is the angular component of it.
So $\dot{\theta}$ has dimension $T^{-1}$.\\
As for accelerations, we have
$$\underline{\ddot{r}}=\underline{\dot{v}}=\ddot{r}\underline{e_r}+\dot{r}\underline{e_\theta}\dot{\theta}+(\dot{r}\dot{\theta}+r\ddot{\theta})\underline{e_\theta}-r\dot{\theta}\underline{e_r}\dot{\theta}=(\ddot{r}-r\dot{\theta}^2)\underline{e_r}+(2\dot{r}\dot{\theta}+r\ddot{\theta})\underline{e_\theta}$$
\begin{example}
    Consider the circular motion with a constant angular velocity, then $r=a,\dot{\theta}=\omega$, so $\dot{r}=\ddot{\theta}=0$, hence
    $$\underline{\ddot{r}}=(\ddot{r}-r\dot{\theta}^2)\underline{e_r}+(2\dot{r}\dot{\theta}+r\ddot{\theta})\underline{e_\theta}=-a\omega^2\underline{e_r}=-\omega^2\underline{r}$$
    which is the familiar centripetal acceleration.
    Newton's Second Law requires a force to be applied to cause this acceleration, which is called the centripetal force, which is in the direction of $-\underline{e_r}$.
    For the special case that it is actually a mass on a string, then when the string broke, the mass will move in a straight line that is tangential to the point where the string broke.
\end{example}
\subsection{Motion in a Constant Force Field}
We know
$$m\underline{\ddot{r}}=\underline{F}=-\nabla V(r)=-\frac{\mathrm dV}{\mathrm dr}\underline{e_r}$$
for a force field that is symmetric wrt the origin.
Note that
$$-\frac{\mathrm dV}{\mathrm dr}\underline{e_r}=\underline{F}=m(\ddot{r}-r\dot{\theta}^2)\underline{e_r}+m(2\dot{r}\dot{\theta}+r\ddot{\theta})\underline{e_\theta}$$
By looking at the $\underline{e_\theta}$ component, we have $2\dot{r}\dot{\theta}+r\ddot{\theta}=0$, so
$$\frac{1}{r}\frac{\mathrm d}{\mathrm dt}(mr^2\dot{\theta})=0$$
Hence the quantity $mr^2\dot{\theta}$ is constant, but $\underline{L}=\underline{r}\times (m\underline{\dot{r}})=mr^2\dot{\theta}\underline{e_z}$.
Where $\underline{e_z}$ is the normal to the plane of motion.
Hence the angular momentum is constant in magnitude.
We write $h=|\underline{L}|/m=r^2\dot{\theta}$.\\
Going to the radial part $\mathrm dV/\mathrm dr=-m(\ddot{r}-h^2/r^3)$, rearranging gives
$$m\ddot{r}=-\frac{\mathrm dV}{\mathrm dr}+\frac{mh^2}{r^3}=-\frac{\mathrm dV_{\rm eff}}{\mathrm dr},V_{\rm eff}=V+\frac{mh^2}{2r^2}$$
So the motion of the particle is as if we are considering one dimensional motion under the influence of a modified potential $V_{\rm eff}$.\\
The energy of the particle is then
$$E=T+V=\frac{1}{2}m|\underline{\dot{r}}|^2+V(r)=\frac{1}{2}m\dot{r}^2+V_{\rm eff}(r)$$
\begin{example}
    For gravity, we have
    $$V(r)=-\frac{GMm}{r},V_{\rm eff}(r)=-\frac{GMm}{r}+\frac{mh^2}{2r^2}$$
    So $V_{\rm eff}$ is minimum at $r=h^2/GM$ and minimum energy is
    $$E_{\rm min}=-m(GM)^2/(2h^2)$$
    At the minimum (which is a stable equilibrium), both $r,\dot\theta$ are constants.\\
    At any $E_{\rm min}<E<0$, the particle oscillates.
    Let $r_0$ be the point where $V_{\rm eff}=0$, then $r_0<r_{\rm min}\le r\le r_{\rm max}$.
    So it gives a bounded non-circular orbit with $\dot\theta$ varying when $r$ varies.
    $r_{\rm min}$ is called the periapsis and $r_{\rm max}$ is called apoapsis.\\
    For $E>0$, the particle can move from a long distance and escape, which is an unbounded orbit.
\end{example}
\subsection{Stability of Circular Orbits}
Consider the potential $V(r)$, we want to investigate whether a circular orbit exists and whether it is stable.\\
Assume that the angular momentum $h$ is given and is nonzero.
For circular orbit $r(t)=r_\star$ is a constant, then
$$\ddot{r}=0\implies V_{\rm eff}^\prime(r_\star)=0$$
which is the condition for circular orbit.
Note that if $V_{\rm eff}^{\prime\prime}(r_\star)>0$, then it is a minimum, so $r_\star$ is a stable fixed point.
So if we express it as $V(r)$, we get
$$0=V_{\rm eff}^\prime(r_\star)=V^\prime(r_\star)-\frac{mh^2}{r_\star^3}=0\implies V^\prime(r_\star)=\frac{mh^2}{r_\star^3}$$
And it is stable if
$$0<V_{\rm eff}^\prime(r_\star)=V^{\prime\prime}(r_\star)+\frac{3mh^2}{r_\star^4}=V^{\prime\prime}(r_\star)+\frac{3V^\prime(r_\star)}{r_\star}$$
So in terms of $F(r)$, we have
$$F^{\prime}(r_\star)+\frac{3F(r_\star)}{r_\star}<0$$
\begin{example}
    If we take $V(r)=-km/rp$ for $k,p>0$ for a circular orbit with radius $r_\star$, we can solve the above equation to get $r_\star=(pk/h^2)^{1/(p-2)}$.
    So unless $p=2$, there exists a circular orbit.\\
    As for stability, we have
    $$V^{\prime\prime}(r_\star)+\frac{3V^\prime(r_\star)}{r_\star}=\frac{p(2-p)k}{r_\star^{p+2}}>0$$
    which is positive iff $p<2$.
\end{example}
\subsection{The Orbit Equation}
The shape of the orbit is obviously governed by the joint variation of $r$ and $\theta$ (both as functions of $t$).
In principle, the energy equation can be helpful to determine $r(t)$, i.e.
$$E=\frac{1}{2}m\dot{r}^2+V_{\rm eff}(r)\implies t=\pm\sqrt{\frac{m}{2}}\int\frac{\mathrm dr}{\sqrt{E-V_{\rm eff}(r)}}$$
Given $r(t)$, since we already know the conservation of angular momentum $r^2\dot\theta=h$, we can then deduce $\theta(t)$.
But this might not always yield an analytic solution.\\
An interesting approach if one is only interested in the trajectory is to use $\theta$ as the dependent variable.
We can write
$$\frac{\mathrm d}{\mathrm dt}=\dot\theta\frac{\mathrm d}{\mathrm d\theta}=\frac{h}{r^2}\frac{\mathrm d}{\mathrm d\theta}$$
So plugging in Newton's Second Law,
$$m\frac{h}{r^2}\frac{\mathrm d}{\mathrm d\theta}\left( \frac{h}{r^2}\frac{\mathrm dr}{\mathrm d\theta} \right)-\frac{mh^2}{r^3}=F(r)$$
which then becomes, by substituting $u=1/r$,
$$\frac{\mathrm d^2u}{\mathrm d\theta^2}+u=-\frac{1}{mh^2u^2}F\left(\frac{1}{u}\right)$$
This is called the orbit equation.
We can then solve for $u$ as a function of $\theta$, and then $\dot\theta=hu^2$ can help us to deduce the time evolution.
\subsection{The Kepler Problem}
We want to solve the case for gravitational central force given by
$$F(r)=-\frac{mk}{r^2}$$
So the orbit equation becomes
$$\frac{\mathrm d^2u}{\mathrm d\theta^2}+u=\frac{k}{h^2}$$
Which is linear in $u$.
We know how to solve this.
Indeed, the general solution is given by
$$u=\frac{k}{h^2}+A\cos(\theta-\theta_0)$$
WLOG we assume $A\ge 0$.\\
If $A=0$, then $u$ is constant hence we obtain a circular orbit.
If $A>0$, $u$ obtains its maximum (hence $r$ obtains its minimum) at $\theta=\theta_0$.
We may choose $\theta_0=0$, then
$$r=\frac{1}{u}=\frac{\ell}{1+e\cos\theta},\ell=\frac{h^2}{k},e=\frac{Ah^2}{k}$$
Which is the polar coordinate form of a conic section with focus at the origin.
$e$ is called the eccentricities, which determines the shape of the trajectory.
By rearranging we obtain (since $r=\ell-ex$ and $r\cos\theta=y$)
$$(1-e^2)x^2+2elx+y^2=\ell^2$$
Therefore if $e\in[0,1)$, it is an ellipse that is bounded by
$$\frac{\ell}{1+e}\le r\le\frac{\ell}{1-e}$$
Or analytically we can rewrite the equation as
$$\frac{(x+ea)^2}{a^2}+\frac{y^2}{b^2}=1,a=\frac{\ell}{1-e^2},b=\frac{\ell}{\sqrt{1-e^2}}\le a$$
$a,b$ represents the semimajor and semiminor axes respectively.
In particular, for $e=0$, the path is a circle with center being the central mass.\\
For $e>1$, the equation gives a hyperbola, so $r\to\infty$ when $\theta\to\pm\alpha$ where $\alpha=\cos^{-1}(-1/e)\in (\pi/2,\pi)$.
We can also transform the equation ot the standard equation for hyperbola
$$\frac{(x-ea)^2}{a^2}-\frac{y^2}{b^2}=1$$
with $a=\ell/(e^2-1),b=\ell/\sqrt{e^2-1}$.
This case represents incoming body with large velocity which is deflected by gravitational force.
By simple calculations the asymptotes are $y=\mp b(x-ea)/a$, so $bx\pm ay=eba$.
And the normal vectors are $\underline{n}=(b,\pm a)/\sqrt{a^2+b^2}$.\\
Now consider the perpendicular distance between incoming mass and orgin, we have
$$\underline{r}\cdot\underline{n}=(x,y)\cdot\left( \frac{b}{\sqrt{a^2+b^2}},\pm\frac{a}{\sqrt{a^2+b^2}} \right)=\frac{eba}{\sqrt{a^2+b^2}}=b$$
This is sometimes called the impact parameter.\\
The marginal case that $e=1$ yields a parabola with equation
$$r=\frac{\ell}{1+\cos\theta}$$
where $r\to \infty$ as $\theta\to\pm\pi$.
In Cartesians this reduces to $y=2\ell(\ell-x)$.\\
On the other hand, we might want to analyze the linkge between the energy and the eccentricity of the trajectory.
Recall that
\begin{align*}
    E&=\frac{1}{2}m(\dot{r}^2+r^2\dot{\theta}^2)-\frac{mk}{r}\\
    &=\frac{1}{2}mh^2\left( \left( \frac{\mathrm du}{\mathrm d\theta} \right)^2+u^2 \right)-mku\\
    &=\frac{mk}{2\ell}(e^2-1)
\end{align*}
Hence bounded orbits have $e<1,E<0$ and unbounded ones have $e>1,E>0$.
The marginal case is then $e=E=0$.
\begin{law}[Kepler's Laws of Planetary Motion]
    1. Orbit of planet is ellipe with the Sun at focus.\\
    2. Line between the planet and the sun sweeps cut equal area in equal time.\\
    3. Square of period $P$ is proportional to cube of semimajor axis.
\end{law}
1 is consistent with the solutrion to the orbit equation that we have obtained earlier, and $2$ follows from the conservation of angular momentum (since the rate of change of area is approximately $r^2\dot\theta/2=h/2$).
Hence the area of the ellipse is $A=hP/2$ where $P$ is the period.
Therefore $\pi ab=hP/2$, rearranging gives the third statement.
\subsection{Rutherford Scattering}
Consider the motion in a repulsive force under inverse square law:
$$V(r)=\frac{mk}{r},F(r)=\frac{mk}{r^2}$$
Then the orbit equation solves to give
$$\frac{1}{r}=u=-\frac{k}{h^2}+A\cos(\theta-\theta_0)$$
WLOG $\theta_0=0,A\ge 0$.
So
$$r=\frac{\ell}{e\cos\theta-1},\ell=\frac{h^2}{k},e=\frac{Ah^2}{k}$$
If there is sometime where $r>0$, then we necessarily have $e>1$, therefore the trajectory is a hyperbola.
As previously known, $r\to\infty$ as $\theta\to\pm\alpha$ where $\alpha=\cos^{-1}(1/e)\in(0,\pi/2)$ and in Cartesian,
$$\frac{(x-ea)^2}{a^2}-\frac{y^2}{b^2}=1,a=\frac{\ell}{e^2-1},b=\frac{\ell}{\sqrt{e^2-1}}$$
Suppose the speed of the particle from far away be $v$, that is
With $x$-axis parallel to the incoming asymptote, as $t\to-\infty$
$$\underline{r}(t)\to (x(t),b,0),\underline{\dot{r}}(t)\to(-v,0,0)$$
Then $\underline{r}\times\underline{\dot{r}}\to(0,0,bv)$
Therefore the angular momentum per unit mass is $bv$, so
$$b=\frac{h^2}{k}\frac{k}{bv^2}=\frac{h^2}{k}\tan\frac{\beta}{2}=\frac{b^2v^2}{k}\tan\frac{\beta}{2},\beta=2\tan^{-1}\left( \frac{k}{bv^2} \right)$$
Rutherfold (1911) fired $\alpha$ particles at gold leaf to obtain experimental results of the scattering.
But Scattering angles greater than $\pi/2$ is observed in the experiment, from which he concluded that the positive charge must be highly concentrated.
    \section{Rotating Frames of Reference}
Newton's Second Law works only in inertial frames.
A rotating frame of reference (wrt an inertial frame) is clearly non-inertial in general.
So the equation of motion in this frame needs to be modified relative to Newton's Second Law.\\
Let $S$ be an inertial frame and $S'$ another frame that is rotating along $z$-axis in $S$ with angular velocity $\omega=\dot\theta$ where $\theta$ is the angle between $x,y$-axis in $S$ and in $S'$.
Denote the basis vectors of $S$ by $\underline{e_1}=\underline{\hat{x}},\underline{e_2}=\underline{\hat{y}},\underline{e_3}=\underline{\hat{z}}$ and that of $S'$ by $\underline{e_1'}=\underline{\hat{x}}',\underline{e_2'}=\underline{\hat{y}}',\underline{e_3'}=\underline{\hat{z}}'$.
Consider a particle at rest in $S'$ viewed in $S$, then its velocity will be
$$\left( \frac{\mathrm d\underline{r}}{\mathrm dt} \right)_S=\underline{w}\times\underline{r}=\omega\underline{\hat{z}}\times\underline{r}$$
Conventionally we take $\omega>0$ as anticlockwise.
We certainly have some formula that applies to the basis vectors of $S'$, namely
$$\left( \frac{\mathrm d}{\mathrm dt}\underline{e_i'} \right)_S=\underline{\omega}\times\underline{e_i'}$$
So for a time-dependent vector $\underline{a}$ we have
$$\underline{a}(t)=\sum_{i=1}^3a_i'(t)\underline{e_i'}(t)$$
So when we observe in $S'$, the rate of change has
$$\left( \frac{\mathrm d}{\mathrm dt}\underline{a}(t) \right)_{S'}=\sum_{i=1}^3\left( \frac{\mathrm d}{\mathrm dt}a_i'(t) \right)\underline{e_i'}(t)$$
Therefore
\begin{align*}
    \left(\frac{\mathrm d}{\mathrm dt}\underline{a}\right)_S&=\sum_{i=1}^3\frac{\mathrm da_i'}{\mathrm dt}\underline{e_i'}+\sum_{i=1}^3a_i'\left( \frac{\mathrm de_i'}{\mathrm dt} \right)_S\\
    &=\sum_{i=1}^3\frac{\mathrm da_i'}{\mathrm dt}\underline{e_i'}+\sum_{i=1}^3a_i'(\underline{\omega}\times\underline{e_i'})\\
    &=\left( \frac{\mathrm d}{\mathrm dt}\underline{a} \right)_{S'}+\underline{\omega}\times\underline{a}
\end{align*}
Which is the key identity that relates rate of change in one frame to that in the other.
If we apply this to the position vector $\underline{r}$, then
$$\left(\frac{\mathrm d}{\mathrm dt}\underline{r}\right)_S=\left( \frac{\mathrm d}{\mathrm dt}\underline{r} \right)_{S'}+\underline{\omega}\times\underline{r}$$
And applying to velocity,
\begin{align*}
    \left( \frac{\mathrm d^2\underline{r}}{\mathrm dt^2} \right)
    &=\left( \left( \frac{\mathrm d}{\mathrm dt} \right)_{S'}+\underline{\omega}\times \right)\left( \left( \frac{\mathrm d}{\mathrm dt}\underline{r} \right)_{S'}+\underline{\omega}\times\underline{r} \right)\\
    &=\left( \frac{\mathrm d^2\underline{r}}{\mathrm dt^2} \right)_{S'}+2\underline{\omega}\times\left( \frac{\mathrm d\underline{r}}{\mathrm dt} \right)_{S'}+\underline{\dot{\omega}}\times\underline{r}+\underline{\omega}\times(\underline{\omega}\times\underline{r})
\end{align*}
This gives the acceleration.
\subsection{Equation of Motion in a Rotating Frame}
$S$ is inertial, therefore Newton's Laws of Motion applies, hence
$$m\left( \frac{\mathrm d^2\underline{r}}{\mathrm dt^2} \right)_S=\underline{F}$$
Hence we have
$$m\left( \frac{\mathrm d^2\underline{r}}{\mathrm dt^2} \right)_{S'}=\underline{F}-m\left( 2\underline{\omega}\times\left( \frac{\mathrm d\underline{r}}{\mathrm dt} \right)_{S'}+\underline{\dot{\omega}}\times\underline{r}+\underline{\omega}\times(\underline{\omega}\times\underline{r}) \right)$$
The second term is known as the fictitious forces, whcih are needed to explain the motion observed in a non-intertial frame.
We give names to each term in the fictitious forces:\\
Coriolis force: $-2m\underline{\omega}\times(\mathrm d\underline{r}/\mathrm dt)_{S'}$.\\
Euler force: $-m\underline{\dot{\omega}}\times\underline{r}$.\\
Centrifugal force: $-m\underline{\omega}\times(\underline{\omega}\times\underline{r})$.
Sometimes we take $\underline{\omega}$ to be constant, so the Euler force will be zero.
\subsection{Centrifugal Force}
Note that for $\underline{\omega}=\omega\underline{\hat{\omega}}$ with $\underline{\hat{\omega}}$ being unit,
\begin{align*}
    -m\underline{\omega}\times(\underline{\omega}\times\underline{r})&=-m((\underline{\omega}\cdot\underline{r})\omega-|\underline{\omega}|^2\underline{r})\\
    &=m\omega^2(\underline{r}-\underline{\hat{\omega}}(\underline{\hat{\omega}}\cdot\underline{r}))\\
    &=m\omega^2\underline{r}_\perp
\end{align*}
where $\underline{r}_\perp$ is the projection of $\underline{r}$ onto the plane that is perpendicular to $\underline{\omega}$, that is basically the plane of rotation.
So the centrifugal force is directed away from the rotatrion axis and its magnitude is $m\omega^2d$ where $d$ is the distance of the particle to the rotation axis.
Note that
$$|\underline{r}_\perp|^2=|\underline{r}|^2-(\underline{\hat{\omega}}\cdot\underline{r})^2=|\underline{r}\times\underline{\hat\omega}|^2$$
While we also have $\nabla{|\underline{r}_\perp|^2}=2\underline{r}-2\underline{\hat\omega}(\underline{\hat\omega}\cdot\underline{r})=2\underline{r}_\perp$.
Therefore
$$m\omega^2\underline{r}_\perp=\nabla\left( \frac{1}{2}m|\underline{r}\times\underline{\omega}|^2 \right)$$
Therefore the centrifugal force is a potential force.
On a rotating planet, we can combine the centrifugal force with gravitational force to create the notion of an effective gravity $\underline{g}_{\rm eff}=\underline{g}+\omega^2\underline{r}_\perp$.
Consider a point $P$ on the surface of the rotating planet, where the rotation axis is through the poles.
We define a local coordinate at $P$ where $\underline{\hat{z}}$ is the normal pointing outwards, $\underline{\hat{y}}$ is tangent northward, and $\underline{\hat{x}}$ is the tangent eastward.
Assume that the point $P$ is at latitude $\lambda$.
So $\underline{r}=R\underline{\hat{z}}$ where $R$ is the radius of the planet.
Also, as for the angular velocity, $\underline{\omega}=\omega(\underline{\hat{y}}\cos\lambda+\underline{\hat{x}}\sin\lambda)$.
\begin{align*}
    \underline{g}_{\rm eff}&=-g\underline{\hat{z}}+\omega^2R\cos\lambda(\underline{\hat{z}}\cos\lambda-\underline{\hat{y}}\sin\lambda)\\
    &=-(g-\omega^2R\cos^2\lambda)\underline{\hat{z}}-\omega^2R\cos\lambda\sin\lambda\underline{\hat{y}}
\end{align*}
So the angle between $\underline{g}$ and $\underline{g}_{\rm eff}$ would be
$$\alpha=\tan^{-1}\left( \frac{\omega^2R\cos\lambda\sin\lambda}{g-\omega^2R\cos^2\lambda} \right)$$
For earth, $\omega\approx 2\pi/86400$, so upon calculation, we obtain $\alpha\approx 3.5\times 10^{-3}$ which is very small.
\subsection{The Coriolis Force}
The coriolis force
$$-2m\underline{\omega}\times(\mathrm d\underline{r}/\mathrm dt)_{S'}=-2m\underline{\omega}\times\underline{v}$$
is perpendicular to the velocity, so it does not do any work.
This is just like the magnetic force.
We consider te horizontal motion on a rotating planet again.
The velocity is given by $\underline{v}=v_x\underline{\hat{x}}+v_y\underline{\hat{y}}$.
As before in our choice of model we have $\underline{\omega}=\omega(\underline{\hat{y}}\cos\lambda+\underline{\hat{z}}\sin\lambda)$.
Therefore
\begin{align*}
    -2m\underline{\omega}\times\underline{v}&=-2m\omega(\underline{\hat{y}}\cos\lambda+\underline{\hat{z}}\sin\lambda)\times v_x\underline{\hat{x}}+v_y\underline{\hat{y}}\\
    &=2m\omega\sin\lambda(v_y\underline{\hat{x}}-v_x\underline{\hat{y}})+2m\omega\cos\lambda v_x\underline{\hat{z}}
\end{align*}
So by considering tthe sign, the horizontal coriolis force gives a acceleration, which is to the right if we are on the northern hemisphere, and to the left on the southern hemisphere.
In atmosphere, the coriolis force can be balanced by a pressure gradient.
The horizontal motion then gives the difference in the direction of cyclones, which is anticlockwise in northern hemisphere and clockwise in the southern hemisphere.
\begin{example}
    Consider a ball dropped from the top of a tower, we want to know where does it land.
    We have
    $$\underline{\ddot{r}}=\underline{g}-2\underline{\omega}\times\underline{\dot{r}}-\underline{\omega}\times(\underline{\omega}\times\underline{r})=\underline{g}-2\underline{\omega}\times\underline{\dot{r}}+O(\omega^2)$$
    where the rotation is slow (i.e. $\omega^2R/g$ is small).
    Integrate it to get
    $$\underline{\dot{r}}=\underline{g}t-2\underline{\omega}\times (\underline{r}-\underline{r}(0))+O(\omega^2)$$
    We substitute this back to the original equation to get $\underline{\ddot{r}}=\underline{g}-2\underline{\omega}\times(\underline{g}t)+O(\omega^2)$, which solves to
    $$\underline{r}=\underline{r}(0)+\frac{1}{2}\underline{g}t^2-\underline{\omega}\times\underline{g}\frac{t^3}{3}+O(\omega^2)$$
    So if we take $\underline{g}=(0,0,-g),\underline{\omega}=(0,\omega,0)$ and $\underline{r}(0)=(0,0,R+h)$, then
    $$\underline{r}=\left(\frac{1}{3}\omega gt^3,0,R+h-\frac{1}{2}gt^2\right)$$
    So the time to reach the ground would be $t=\sqrt{2h/g}$, then it would travel a horizontal distance of approximately
    $$\frac{1}{3}\omega g\left(\frac{2h}{g}\right)^{3/2}$$
\end{example}
(Foucaul Pendulum)
Consider a pendulum at north pole, then the plane of its oscillation is rotating opposing the direction of rotation of the earth.
At latitude $\lambda$, the angular velocity of plane of rotation is $\omega\sin\lambda$, therefore the period $2\pi/(\omega\sin\lambda)$ which is greater than a day if $\lambda<\pi/2$.
    \section{System of Particles}
We have considered the motion of a single particle in a force field, so we will now turn to a system of particles where they act on each other.\\
Consider $N$ particles, namely particles $1,\ldots,i,\ldots,N$ having masses $m_i$ and positions $\underline{r_i}(t)$.
The momentums are then $\underline{p_i}=m_i\underline{\dot{r}_i}$.
Newton's Second Law for one of these particles is then $m_i\underline{\ddot{r}_i}=\underline{\dot{p}_i}=\underline{F_i}$.
Divide the forces exerted in two parts: the external forces $\underline{F_i}^{\rm ext}$ (causes by something outside the $N$ particles) and internal forces $\underline{F_{ij}}$ which is the force exerted by particle $j$ on $i$.
Conventionally we take $\underline{F_{ii}}=\underline{0}$.
So basically
$$\underline{F_i}=\underline{F_i}^{\rm ext}+\sum_j\underline{F_{ij}}$$
Newton's Third Law then tells us $\underline{F_{ij}}+\underline{F_{ji}}=0$, like gravitation.
\subsection{Motion of the Centre of Mass}
The total mass of the system is $M=\sum_im_i$, then we define the centre of mass to be a location
$$\underline{R}=\frac{1}{N}\sum_{i=1}^Nm_i\underline{r_i}$$
The total linear momentum would be $\underline{P}=\sum_i\underline{p_i}=\sum_im_i\underline{\dot{r}}=M\underline{\dot{R}}$.
Consider the rate of change of the momentum.
By Newton's Secon Law,
$$\underline{\dot{P}}=M\underline{\dot{R}}=\sum_{i=1}^N\underline{\dot{p}_i}=\sum_{i=1}^N\underline{F_i}^{\rm ext}+\sum_{i=1}^N\sum_{j=1}^N\underline{F_{ij}}=\sum_{i=1}^N\underline{F_i}^{\rm ext}=\underline{F}^{\rm ext}$$
By $\underline{F_{ij}}=-\underline{F_{ji}}$.
So the motions of the centre of mass closely resembles that of a single particle with mass $M$ and position $\underline{R}$, which is reassuring since it means we can consider finite (bounded) bodies as particles.
So Newton's Second Law applies to macroscopic objects.
A simple conclusion from this is if $\underline{F}^{\rm ext}=0$, then the total momentum of the system $\underline{\dot{P}_i}=\sum_{i=1}^N\underline{\dot{p}_i}$ is conserved.
So in this case, we can set up the inertial frame as the centre of mass frame, in which $\underline{\dot{R}}=\underline{0}$.\\
Now we turn to consider the angular momentum.
Consider the total angular momentum about the origin, which is $\underline{L}=\sum_i\underline{r_i}\times\underline{p_i}$, then
\begin{align*}
    \underline{\dot{L}}&=\sum_{i=1}^N\underline{r_i}\times\underline{\dot{p}_i}+\sum_{i=1}^N\underline{\dot{r}_i}\times\underline{p_i}\\
    &=\sum_{i=1}^N\underline{r_i}\times\underline{\dot{p}_i}\\
    &=\sum_{i=1}^N\underline{r_i}\times\underline{F_i}^{\rm ext}+\sum_{i=1}^N\sum_{j=1}^N\underline{r_i}\times\underline{F_{ij}}\\
    &=\underline{G}^{\rm ext}+\frac{1}{2}\sum_{i=1}^N\sum_{j=1}^N(\underline{r_i}-\underline{r_j})\times\underline{F_{ij}}
\end{align*}
The last term is sometimes zero, in which case the rate of change would be $\underline{G}^{\rm ext}$, the total external torque.
\subsection{Motion relative to the Centre of Mass}
Write $\underline{r_i}=\underline{R}+\underline{s_i}$, so $\underline{s_i}$ is the position of particle $i$ relative to the centre of mass.
Then
$$\sum_{i=1}^Nm_i\underline{s_i}=\sum_{i=1}^Nm_i(\underline{r_1}-\underline{R})=\sum_{i=1}^Nm_i\underline{r_i}-M\underline{R}=0$$
Consequently $\sum_im_i\underline{\dot{s}_i}=0$.
As for the total linear momentum, since we have the above,
$$\underline{P}=\sum_{i=1}^Nm_i(\underline{\dot{R}}+\underline{\dot{s}_i})=M\underline{\dot{R}}$$
The angular momentum would have, exploiting the same fact,
\begin{align*}
    \underline{L}&=\sum_{i=1}^Nm_i(\underline{R}+\underline{s_i})\times(\underline{\dot{R}}+\underline{\dot{s}_i})\\
    &=\sum_{i=1}^Nm_i\underline{R}\times\underline{\dot{R}}+\sum_{i=1}^Nm_i\underline{s_i}\times\underline{\dot{s}_i}+\left( \sum_{i=1}^Nm_i\underline{s_i} \right)\times\underline{\dot{R}}+\underline{R}\times\left( \sum_{i=1}^Nm_i\underline{\dot{s}_i} \right)\\
    &=\sum_{i=1}^Nm_i\underline{R}\times\underline{\dot{R}}+\sum_{i=1}^Nm_i\underline{s_i}\times\underline{\dot{s}_i}\\
    &=M\underline{R}\times\underline{\dot{R}}+\sum_{i=1}^Nm_i\underline{s_i}\times\underline{\dot{s}_i}
\end{align*}
which is the angular momentum of the centre of mass plus the total angular momentum relative to the centre of mass.
The total kinetic energy is
\begin{align*}
    T&=\sum_{i=1}^N\frac{1}{2}m_i|\underline{\dot{r}_i}|^2\\
    &=\sum_{i=1}^N\frac{1}{2}m_i|\underline{\dot{R}}+\underline{\dot{s}_i}|^2\\
    &=\sum_{i=1}^N\frac{1}{2}m_i|\underline{\dot{R}}|^2+\sum_{i=1}^N\frac{1}{2}m_i|\underline{\dot{s}_i}|^2+\underline{\dot{R}}\cdot\left(\sum_{i=1}^Nm_i\underline{\dot{s}_i}\right)\\
    &=\frac{1}{2}M|\underline{\dot{R}}|^2+\frac{1}{2}\sum_{i=1}^Nm_i|\underline{\dot{s}_i}|^2
\end{align*}
Now is the energy conserved?
Assuming $\underline{F}^{\rm ext}$ is conserved, then $\underline{F}^{\rm ext}=-\nabla V^{\rm ext}$.
Take $\underline{F_{ij}}$ to be conserved as well and its potential purely depends on the seperation between the particles, then $\underline{F_{ij}}=-\nabla V_{ij}(\underline{r_i}-\underline{r_j})$.
We can show that the total energy is conserved under these assuming these by just differentiating.
\subsection{The Two Body Problem}
The centre of mass is $\underline{R}=(m_1\underline{r_1}+m_2\underline{r_2})/M$
Consider the seperation vector $\underline{r}=\underline{r_1}-\underline{r_2}$, so we can write
$$\begin{cases}
    \underline{r_1}=\underline{R}+\frac{m_2}{m}\underline{r}\\
    \underline{r_2}=\underline{R}-\frac{m_1}{m}\underline{r}
\end{cases}$$
Since the external force is assumed to be zero, $\underline{R}$ moves with constant velocity.
Consider $\ddot{r}$, then $\ddot{r}=\underline{F_{12}}/m_1-\underline{F_{21}}/m_2=(1/m_1+1/m_2)\underline{F_{12}}$.
Hence
$$\mu\underline{\ddot{r}}=\underline{F_{12}}(\underline{r}),\mu=\frac{m_1m_2}{m_1+m_2}$$
Here $\mu$ is called the reduced mass.
In the case of the gravitational force, we have
$$\mu\underline{\ddot{r}}=-\frac{Gm_1m_2}{|\underline{r}|^3}\underline{r}\implies\underline{\ddot{r}}=-G(m_1+m_2)\frac{\underline{r}}{|\underline{r}|^3}=-\frac{GM}{|\underline{r}|^3}\underline{r}$$
which is just like the motion of two particles entirely due to a mass $M$ at the origin.
So both masses perform orbits with similar shape but different sizes.
\subsection{Variable Mass Problem}
Think of a rocket whose mass decreases as it moves due to the exhausted mass.
So its mass itself is variable.
So we need to apply Newton's Second Law to the whole system including the exhausted mass.
Suppose the mass and the velocity of the rocket is $m(t),v(t)$.
And the exhausted mass has a speed $u$ relative to the rocket when leaving the rocket.
At time $t$, when we look at this instant only, we can ignore what happened in the past, hence after a small time interval $\delta t$, mass of $m(t)=m(t+\delta t)$ is exhausted with speed $v(t)-u+o(\delta t)$.
By conservation of momentum,
$$m(t+\delta t)v(t+\delta t)+(m(t)-m(t+\delta t))(v(t)-u+o(\delta t))=m(t)v(t)$$
Hence
$$(mv^\prime+m^\prime u)\delta t\approx m(t+\delta t)(v(t+\delta t)-v(t))+(m(t+\delta t)-m(t))u+o(\delta t)=0$$
Since $\underline{F}=\underline{\dot{p}}$, it generalises to
$$mv^\prime+m^\prime u=F$$
where $F$ is the total external force exerted on the rocket.
This is called the rocket equation.
If $F=0$, we have $mv^\prime+m^\prime u=0$ which we can solve to get $v=v_0+u\log(m_0/m(t))$.
    \section{Rigid Bodies}
\begin{definition}
    A rigit body is an extended mass with a finite volume as a system of particles that are constrained such that the mutual distances between them does not change.
\end{definition}
\begin{definition}
    An isometry is a distance-preserving map in the space, e.g. rotation, translation, etc..
\end{definition}
So a rigid body is a system of particle moving under isometries.
\subsection{Angular Velocity}
Recall that we can have a vector angular velocity $\underline{\omega}$ which points to the axis of rotation and has magnitude equal to the scalar angular velocity $\omega$ of the point mass $\underline{r}$.
So $\underline{\dot{r}}=\underline{\omega}\times\underline{r}$.
If the particle has mass $m$, we can write down the kinetic energy $T=m|\underline{\dot{r}}|^2/2=m\omega^2r_\perp^2/2$ where $r_\perp=|\underline{n}\times\underline{r}|$ where $\underline{n}$ is a unit vector and $\underline{\omega}=\omega\underline{n}$.
We write $I=mr_\perp^2$ as the moment of inertia, so $T=I\omega^2/2$.
Note that the moment of inertia is dependent on the axis.
\subsection{Moment of Inertia for a Rigid Body}
Consider a rigid body made up of $N$ particles following the notation we introduced earlier.
The body (i.e. all the particles within it) would rotate about an axis through the origin with angular velocity $\underline{\omega}$.
For particle $i$, we have $\underline{\dot{r}_i}=\underline{\omega}\times\underline{r_i}$.
Note that
$$\frac{\mathrm d}{\mathrm dt}|\underline{r_i}-\underline{r_j}|^2=2((\omega\times(\underline{r_i}-\underline{r_j}))\cdot(\underline{r_i}-\underline{r_j}))=0$$
So the particles do stay the same distance apart.
The kinetic energy of the rotating body is then going to be
$$T=\sum_{i=1}^N\frac{1}{2}m_i|\underline{\dot{r}_i}|^2=\frac{1}{2}\omega^2\sum_{i=1}^Nm_i|\underline{n}\times\underline{r_i}|^2=\frac{1}{2}\omega^2\sum_{i=1}^Nm_i(r_i)_\perp^2=\frac{1}{2}\omega^2I$$
where $I$ is called the moment of inertia for the body.
Correspondingly, we can consider the angular momentum, where
$$\underline{L}=\sum_{i=1}^N\underline{L_i}=\sum_{i=1}^Nm_i\underline{r_i}\times(\underline{\omega}\times\underline{r_i})=\omega\sum_{i=1}^Nm_i\underline{r_i}\times(\underline{n}\times\underline{r_i})$$
Consider its component in the direction of the axis of rotation,
$$\underline{L}\cdot\underline{n}=\omega\sum_{i=1}^Nm_i\underline{n}\cdot(\underline{r_i}\times(\underline{n}\times\underline{r_i}))=\omega\sum_{i=1}^nm_i|\underline{n}\times\underline{r_i}|^2=I\omega$$
So the direction of $\underline{L}$ in the direction of the axis of rotation is $I\omega$.
In general, $\underline{L}$ is not parallel to $\underline{\omega}$, so we need to go back to the vector expression.
Observe that $\underline{L}$ as a function of $\omega$ is linear, so
$$\underline{L}=\sum_{i=1}^Nm_i\underline{r_i}\times(\underline{\omega}\times\underline{r_i})=\sum_{i=1}^Nm_i(|\underline{r_i}|^2\underline{\omega}-|\underline{r_i}\cdot\underline{\omega}|\underline{r_i})=I\underline{\omega}$$
where $I$ here is a tensor (i.e. a (multi)linear map) which in this case is a $3\times 3$ matrix.
So under suffix notation, $L_\alpha=I_{\alpha\beta}\omega_\beta$.
$I$ is a symmetric tensor (matrix) by symmetry.
We have
$$I_{\alpha\beta}=\sum_{i=1}^Nm_i(|\underline{r_i}|^2\delta_{\alpha\beta})-(\underline{r_i})_\alpha(\underline{r_i})_\beta$$
Now $I$ is diagonalizable so we can choose our favourite basis (principal axes) to make $I$ diagonal.
To get $\underline{L}$ to be at the same direction as $\underline{\omega}$, we need the object to rotate wrt a principal axis
\subsection{Calculation of Moment of Inertia}
For a solid body, we replace mass-weighted sums by mass-weighted volume integrals.
Consider a body with volume $V$ with density $\rho(\underline{r})$, so its mass, center of mass and moment of inertia are
$$M=\int_V\rho\,\mathrm dV,\underline{R}=\frac{1}{M}\int_V\rho(\underline{r})\underline{r}\,\mathrm dV,I=\int_V\rho(\underline{r})|\underline{r}_\perp|^2\,\mathrm dV=\int_V\rho(\underline{r})|\underline{n}\times\underline{r}|^2\,\mathrm dV$$
For curves and surfaces, we can use line and area integrals accordingly.
\begin{example}
    1. For uniform thin ring of mass $M$ and radius $a$ with rotation axis $\underline{n}$ through the center of the ring and perpendicular to the plane where the ring is on.
    In this case, we can reduce volume integral to line integral.
    We have $\rho=M/(2\pi a)$, so
    $$I=\int_0^{2\pi}\left( \frac{M}{2\pi a} \right)a^2a\,\mathrm d\theta=Ma^2$$
    Every point in the body is of the same distance from the axis $|\underline{r}_\perp|=|\underline{n}\times\underline{r}|=a$.\\
    2. Consider a uniform thin rod of mass $M$ and length $l$ with axis of rotation through one end and perpendicular to the rod.
    So
    $$I=\int_0^l\left( \frac{M}{l} \right)x^2\,\mathrm dx=\frac{1}{3}Ml^2$$
    3. Consider a uniform thin disk with mass $M$ and radius $a$ with the axis of rotation through its center and perpendicular to the plane where the disk is in.
    So we use an area integral
    $$I=\int_0^a\int_0^{2\pi}\left( \frac{M}{\pi a^2} \right)r^2r\,\mathrm d\theta\,\mathrm dr=\frac{Ma^2}{2}$$
    4. Using the same disk but choose the axis to be one through the center and in the same plane as the disk.
    In this case,
    $$I=\int_0^a\int_0^{2\pi}\left( \frac{M}{\pi a^2} \right)(r^2\sin^2\theta)r\,\mathrm d\theta\,\mathrm dr=\frac{1}{4}Ma^2$$
    5. Consider a solid sphere (a ball) of mass $M$ and radius $a$ with axis of rotation through its center, so spherical polars will be a good choice.
    We assume WLOG that $\underline{n}$ is the $z$ direction (so $\theta=0$ along $\underline{n}$).
    By uniform density, we have $\rho=3M/(4\pi a^3)$, therefore
    $$I=\int_0^a\int_0^\pi\int_0^{2\pi}\frac{3M}{4\pi a^3}(r^2\sin^2\theta)r^2\sin\theta\,\mathrm d\phi\,\mathrm d\theta\,\mathrm dr=\frac{2}{5}Ma^2$$
\end{example}
There are a few simple but general results to simplify calculation moment of inertia.
\begin{theorem}[Perpendicular Axis Theorem]
    For a two dimensional body on a plane (aka lamina),
    $$I_z=I_x+I_y$$
    where $I_z$ is the moment of inertia along the $z$ axis chosen to be a normal to the plane and $I_x,I_y$ are the moments of inertia along two chosen perpendicular axes on the plane so that all three axes meet at the origin.
\end{theorem}
\begin{proof}
    We have
    $$I_x=\int_A\rho y^2\,\mathrm dA,I_y=\int_A\rho x^2\,\mathrm dA$$
    But
    $$I_z=\int_A\rho r^2\,\mathrm dA=\int_A\rho(x^2+y^2)\,\mathrm dA=I_x+I_y$$
    As desired
\end{proof}
Sometimes the lamina is symmetric enough such that $I_x=I_y$, so $I_z=2I_x$.
This corresponds to the example of a disk.
Note that this theorem works for lamina but does not work for $3$ dimensional bodies.
\begin{theorem}[Parallel Axes Theorem]
    If a rigid body of mass $M$ has moment of inertia $I_c$ about an axis through its center of mass, then for another axis parallel to the original axis with a distance $d$ away, then the moment of inertia $I$ about the new axis is $I=I_c+Md^2$.
\end{theorem}
\begin{proof}
    Choose Cartesian axes such that the centre of mass is at the origin and the rotation axis along $z$-axis.
    Also, choose $x,y$-axes such that the second axes of rotation is through the point $d\underline{\hat{x}}=(d,0,0)$, then
    \begin{align*}
        I_c+Md^2&=\int_V\rho(x^2+y^2)\,\mathrm dV+Md^2\\
        &=\int_V\rho((x-d)^2+y^2)\,\mathrm dV+2d\int_V\rho x\,\mathrm dV\\
        &=\int_V\rho((x-d)^2+y^2)\,\mathrm dV=I
    \end{align*}
    Since the axes are chosen in a way that the origin is the center of mass.
\end{proof}
\begin{example}
    Consider a uniform disk as before with the axis of rotation perpendicular to it through a point on the edge has $I=3Ma^2/2$.
\end{example}
\subsection{Motion of a Rigid Body}
General motion of a rigit body can be described by the composition of translation (of the center of mass) following some trajectory $\underline{R}(t)$ together with a rotation about the center of mass.
Following the previous discussion, we specify points in the body relative to the center of mass by writing $\underline{r_i}=\underline{R}+\underline{s_i}$.
Also recall that $\sum_im_i\underline{r_i}=M\underline{R}$, therefore $\sum_im_i\underline{s_i}=0$.
If a body rotates about its center of mass, with angular velocity $\underline{\omega}$, so $\underline{\dot{s}_i}=\underline{\omega}\times\underline{s_i}$ and $\underline{\dot{r}_i}=\underline{\dot{R}}+\underline{\omega}\times\underline{s_i}$.
The kinetic energy, as we recall, satisfies
$$T=\frac{1}{2}M|\underline{\dot{R}}|^2+\frac{1}{2}\sum_{i=1}^Nm_i|\underline{s_i}|^2=\frac{1}{2}M|\underline{\dot{R}}|^2+\frac{1}{2}I_c\omega^2$$
where $I_c$ is the moment of inertia parallel to $\underline{\omega}$ and through the center of mass.
So $T$ is the sum of translational KE and rotational KE.
We have also shown before that for a general multiparticle system, linear and angular momentum obey $\underline{\dot{P}}=\underline{F},\underline{\dot{L}}=\underline{G}$ where $\underline{F},\underline{G}$ are the total external applied force and torque respectively.
For a rigit body, these two equations determine the translational and rotational motion.
Sometimes, we can exploit the conservation of energy as an easier method of solution.\\
$\underline{L},\underline{G}$ depend on the choice of origin, and we can the origin to be any point fixed in an inertial frame (shown previously).
Or, we can define $\underline{L}$ and $\underline{G}$ about the center of mass, and the equation above, as we shall show, still holds.
Take
\begin{align*}
    \underline{G}&=\frac{\mathrm d}{\mathrm dt}\left(M\underline{R}\times\underline{\dot{R}}+\sum_{i=1}^Nm_i\underline{s_i}\times\underline{\dot{s}_i}\right)\\
    &=M\underline{R}\times\underline{\ddot{R}}+\frac{\mathrm d}{\mathrm dt}\left( \sum_{i=1}^Nm_i\underline{s_i}\times\underline{\dot{s}_i} \right)\\
    &=\underline{R}\times\underline{F}^{\rm ext}+\frac{\mathrm d}{\mathrm dt}\left( \sum_{i=1}^Nm_i\underline{s_i}\times\underline{\dot{s}_i} \right)
\end{align*}
Therefore
\begin{align*}
    \frac{\mathrm d}{\mathrm dt}\left( \sum_{i=1}^Nm_i\underline{s_i}\times\underline{\dot{s}_i} \right)&=\underline{G}-\underline{R}\times\underline{F}^{\rm ext}\\
    &=\sum_{i=1}^N\underline{r_i}\times\underline{F_i}^{\rm ext}-\underline{R}\times\sum_{i=1}^N\underline{F_i}^{\rm ext}\\
    &=\sum_{i=1}^N(\underline{r_i}-\underline{R})\times\underline{F_i}^{\rm ext}\\
    &=\underline{G_c}
\end{align*}
Consider now the motion in a uniform gravitational field with acceleration due to gravity $\underline{g}$, then the total gravitational force and torque acting on a rigit body would be the same as if it is acting on a particle of mass $m$ located in the center of mass (hence it is also called the center of gravity).
So
$$\underline{F}=\sum_{i=1}^N\underline{F_i}^{\rm ext}=\sum_{i=1}^Nm_i\underline{g}=M\underline{g}$$
similarly
$$\underline{G}=\sum_{i=1}^N\underline{G_i}^{\rm ext}=\sum_{i=1}^N\underline{r_i}\times(m_i\underline{g})=M\underline{R}\times\underline{g}$$
Note that the gravitational torque about the center of mass is zero since
$$\underline{G_c}=\sum_{i=1}^N\underline{s_i}\times(m_i\underline{g})=\left( \sum_{i=1}^Nm_i\underline{s_i} \right)\times\underline{g}=0$$
Consider the gravitatioinal potential $-m\underline{r}\cdot\underline{g}$, then
$$V^{\rm ext}=\sum_{i=1}^NV_i^{\rm ext}=\sum_{i=1}^N(-m_i\underline{r_i}\cdot\underline{g})=-M\underline{R}\cdot\underline{G}$$
\begin{example}
    1. Throw a stick in the air.
    So the center of mass follows a parabolic curve and the angular velocity of the stick about center of mass is constant by conservation of energy (or because gravitational torque about center of mass is $0$).\\
    2. A uniform rod of length $l$ and mass $M$ fixed at a pivot point $O$ at one end and makes an angle $\theta$ with the downward vertical.
    We say this is a compound pendulum since the mass is distributed instead of concentrated.
    Consider the angular velocity and angular momentum about the pivot.
    We have $\omega=\dot\theta,L=I\dot\theta=Ml^2\dot\theta/3$.
    So the gravitational torque about $O$ becomes $-Mgl\sin\theta/2$, so $\dot{L}=G\implies I\ddot\theta=-Mgl\sin\theta/2$, so
    $$\ddot\theta=-\frac{3}{2}\frac{g}{l}\sin\theta$$
    which just looks like a simple pendulum of length $2l/3$ (in fact equivalent to it).
    So for small oscillations, the frequency is $f=\sqrt{3g/(2l)}$ and period $2\pi/f$.\\
    Alternatively we can think of the energy, then
    $$E=T+V=\frac{1}{2}I\omega^2-\frac{Mgl}{2}\cos\theta$$
    So
    $$0=\frac{\mathrm dE}{\mathrm dt}=\dot\theta\left(I\ddot\theta+\frac{Mgl}{2}\sin\theta\right)=0$$
    which produces the same result as above.
\end{example}
\subsection{Sliding and Rolling}
Consider a cylinder or sphere with radius $a$ moving along a stationary horizontal surface, then the general motion is a translation of the center of mass with velocity $v$ together with rotation about the center of mass with angular velocity $\omega$.
Let $P$ be the instantaneous point of contact, then the horizontal velocity of this point is given by $v_{\rm slip}=v-a\omega$.\\
There are two extreme cases:\\
1. Pure sliding, where we have $\omega=0,v_{\rm slip}=v\neq 0$.
So the point of contact slips through the surface (probably due to a kinetic frictional force).\\
2. Pure rolling, where we have $\omega,v\neq 0$ but $v_{\rm slip}=v-a\omega=0$.
In this case, the contact point is stationary at any point, which produces rolling without sliding.\\
Instantaneously, we can view the motion of the body as the rotation of the body about the contact point.
Also note that these also apply to inclined plane.
\begin{example}
    Consider a cylinder of radius $a$ and mass $m$ rolling through inclined plane at angle $\alpha$ to the horizontal.
    Let $x$ be the distance down slope travelled by the center of mass, $v=\dot{x}$ and $Mg$ the gravitational force, $N$ the normal reaction and $F$ the frictional force.
    For the cylinder to be purely rolling, we must have $v-a\omega=0$, so $v=a\omega$.\\
    The kinetic energy has
    $$T=\frac{1}{2}Mv^2+\frac{1}{2}I\omega^2=\frac{1}{2}\left(M+\frac{I}{a^2}\right)v^2$$
    Note that due to their directions the normal and frictional force (in the case where $v_{\rm slip}=0$) do no work.
    Now the energy $T+V$ is conserved where $V=-Mgx\sin\alpha$, so
    \begin{align*}
        0&=\frac{\mathrm d(T+V)}{t}\\
        &=\frac{\mathrm d}{\mathrm dt}\left( \frac{M+I/a^2}{2}\dot{x}^2-Mgx\sin\alpha \right)\\
        &=(M+I/a^2)\dot{x}\ddot{x}-Mg\dot{x}\sin\alpha\\
        \implies \left( M+\frac{I}{a^2} \right)\ddot{x}&=Mg\sin\alpha
    \end{align*}
    Note that when $I=0$, this is exactly the equation for a frictionless particle, therefore the rotation makes acceleration smaller.
    Now for a cylinder in question, we have $I=Ma^2/2$, hence
    $$\ddot{x}=\frac{2}{3}g\sin\alpha$$
    We can also obtain the result by using forces and torques.
    By considering the rate of change of linear momentum along the plane, we have $M\dot{v}=Mg\sin\alpha-F$ and the rate of change of angular momentum about the center of mass then gives $I\dot\omega=aF$.
    So as it is rolling, $\dot{v}=a\dot\omega$, whence
    $$M\dot{v}=Mg\sin\alpha-\frac{I\dot{v}}{a^2}$$
    Thus $(M+I/a^2)\dot{v}=Mg\sin\alpha$ as above.\\
    There is yet another way to do this:
    Consider the torque about $P$, we have $I_P=I+Ma^2$ by the parallel axis theroem, also the gravitational torque has $I_P\dot\omega=Mga\sin\alpha$.
    So $v=a\omega$ gives $(I+Ma^2)\dot{v}/a=Mga\sin\alpha$.
\end{example}
\begin{example}
    We want to study the transition from a sliding motion to a rolling one.
    Consider a snooker ball on a horizontal plane hit by a cue instantaneously which gives it an initial velocity $v_0$.
    Initially $v=v_0$ and $\omega_0$, where sliding occurs (so no rotation at $t=0$).
    The kinetic frictional force obeys $F=\mu N=\mu Mg$ where $\mu$ is a constant (coefficient of kinetic friction).
    The linear motion has $M\dot{v}=-F$ and the angular motion $I\dot\omega=aF$.
    Also for a sphere $I=2Ma^2/5$, hence we have, by integrating,
    $$\begin{cases}
        v=v_0-\mu gt\\
        \omega=5\mu gt/(2a)
    \end{cases}$$
    So when the ball is still moving,
    $$0\le v_{\rm slip}=v-a\omega=v_0-\frac{7}{2}\mu gt$$
    So the total time of rolling is $t_{\rm roll}=2v_0/(7\mu g)$.
    During $0\le t\le t_{\rm roll}$, the friction acts to decrease $v$ and increase $\omega$ till the no-slip condition is satisfied, when $t=t_{\rm roll}$ and $v=v_{\rm roll}=5v_0/7$.
    But at $t_{\rm roll}$, the rolling could as well persist but the friction does no further work.
    At $t=t_{\rm roll}$, the kinetic energy is
    $$T=\frac{1}{2}Mv^2+\frac{1}{2}I\omega^2=\frac{1}{2}M\left( 1+\frac{2}{5} \right)v_{\rm roll}^2=\frac{5}{7}\left( \frac{1}{2}Mv_0^2 \right)$$
    So the loss of KE due to friction has a total of
    $$\int_0^{t_{\rm roll}}Fv_{\rm slip}\,\mathrm dt=\int_0^{t_{\rm roll}}F\left( v_0-\frac{7}{2}\mu gt \right)\,\mathrm dt=\frac{1}{7}Mv_0^2$$
\end{example}

    \section{Special Relativity}
The Newtonian Mechanics works perfectly (maybe not) well in low-speed cases, but when the object has gotten a pretty big velocity, Newtonian physics is no longer a good approximation to the situation that arises.
Therefore, in 1905, Albert Einstein proposed the Special Theory of Relativity, in which the main differences involved are due to the treatment of the speed of light $c=299792458{\rm ms^{-1}}\approx 3\times 10^8{\rm ms^{-1}}$.\\
Special Relativity is based on two postulates:
\begin{postulate}[Principle of Relativity]
    The laws of physics are the same in all inertial frames.
\end{postulate}
\begin{postulate}[Speed of Light]
    The speed of light in vacuum is the same in all inertial frames.
\end{postulate}
The need for the second postulate arises from many experiments that failed to detect the dependence of speed of light relative to inertial frames.
But the addition of this postulate then leads to a radical revision of our understanding of space and time and the relationships of energy, momentum and mass.\\
Consider two frames $S,S'$, then if they are related by Galilean transformation, we have
$$x'=x-vt,y'=y,z'=z,t'=t$$
Write the path of light ray in $S$ as $x=ct$, then in $S'$, we have $x'=x-vt=(c-v)t'$, so it doesn't work.
Therefore we need a new form of transformation to describe inertial frames in order to accomodate our postulates.
We have to treat space and time equally.
\subsection{Lorentz Transformation}
Consider inertial frames $S,S'$.
Assume their origins coincide, i.e. the spacial origins of the frames coincide when $t=t'=0$.
Suppose $S'$ is moving along the $x$ direction relative to $S$ with speed $v$, then we can ignore the $y,z$ directions for the moment.
So we are interested in the relationship between $(x,t)$ and $(x',t')$.
By the Principle of Relativity, something moving in constant velocity in $S$ must also do so in $S'$.
In $(x,t)$ plane, the constant velocity path is a straight line, so it is also the case in $(x',t')$.
So the transformation must be linear.
The origin of $S'$ moves with speed $v$ in $S$, this implies that $x'=\gamma(x-vt)$.
where $\gamma$ depends on $|v|$.
By symmetry, $x=\gamma(x'+vt')$.
Consider a light ray going through the origins at time $t=t'=0$.
In $S$, the equation of the light ray in $S$ is $x=ct$ and also in $S'$, $x'=ct'$.
So if we plug these in, then $ct=\gamma(c+v)t'$ and $ct'=\gamma(c-v)t$.
We then have
$$\gamma^2(1-v/c)(1+v/c)=1\implies \gamma=\frac{1}{\sqrt{1-v^2/c^2}}$$
We call $\gamma$ the Lorentz factor.
Consequently we obtain the Lorentz transformation (or Lorentz Boost):
$$\begin{cases}
    x'=\gamma(x-vt)\\
    t'=\gamma(t-vx/c^2)
\end{cases},\begin{cases}
    x=\gamma(x+vt)\\
    t=\gamma(t'+vx'/c^2)
\end{cases}$$
The coordinates $y,z,y',z'$ does not change if the velocity is entirely on the $x$-direction.
Now $\gamma>1$ whenever $v\neq 0$ and $\gamma\to\infty$ if $|v|\to c$.
When $v$ is small, we can approximate $\gamma=1$, which gives us the standard Galilean transformation.\\
To check that the speed of light indeed remains constant in two frames.
Suppose a light ray travels in $x$ direction, then $x=ct$, so
$$x'=\gamma(x-vt)=\gamma(c-v)t=\gamma^2(c-v)\left( t'+\frac{vx'}{c^2} \right)\implies x'=ct'$$
For a light ray that is travelling in the $y$ direction, $y=ct,x=z=0$.
In $S'$, we have
$$x'=\gamma(-vt)=-\gamma vt,t'=\gamma t,y'=ct,z'=0$$
So the speed of the light ray will be
$$\sqrt{\left( \frac{-\gamma t}{\gamma} \right)^2+\left( \frac{c}{\gamma} \right)^2}=\sqrt{c^2}=c$$
So the speed of light is not changed, but the direction of the light ray has.\\
From a more general viewpoint, we consider the metric
$$c^2t'^2-r'^2=(ct')^2-(x'^2+y'^2+z'^2)=(ct)^2-(x^2+y^2+z^2)=c^2t^2-r^2$$
By some calculation.
So this quantity is preserved.\\
Consider the case where there is only one spacial dimension $x$ in an inertial frame $S$ with time $t$.
Conventionally we plot $x$ in the horizontal axis and $ct$ in the vertical direction.
The trajectory of a particle in space-time then is a curve in the plane.
We call this the Minkowski space-time, where each point $(x,ct)$ in the space-time represents an event.
We call the curve that represents the motion of some particle a world line.
In particular, the world line is straight iff the particle moves in uniform velocity.
Light rays through the origin then travels in vertical lines of the form $x=\pm ct$, which are the vertical lines that has a inclination of $\pi/4$ to either axis.
As a particle is not allowed to move with velocity greater than the speed of light, its motion (assuming that it goes through the origin) is restricted to the upper and lower cones that are split by the lines $x=\pm ct$.\\
How about viewing from another inertial frame $S'$?
The $t'$ axis corresponds to $x'=0$, so it corresponds to $x=vt=(v/c)ct$, and the $x'$ axis is $t'=0$, hence the axis is $ct=(v/c)x$.
Thus the axes moves by the same degree closer to the diagonal (where the light ray travels) if $v\ge 0$ and further from the diagonla otherwise.
This is consistent with the postulate that the speed of light doesn't change across the frames.
\subsection{Relativistic Physics}
Consider two events $P_1=(x_1,t_1),P_2=(x_2,t_2)$ that are points in the frame $S$ in one-dimensional Minkowski space-time.
They are called simultaneous if $t_1=t_2$.\\
So the line $P_1P_2$ is then parallel to the $x$-axis.
This is called the line of simultaneity in $S$.
But in $S'$, assuming $v\neq 0$, $P_1,P_2$ are no longer in a line of simultaneity in $S'$ (which is of the form $t-vx/c^2=d$ where $d$ is a constant).
In particular, if $x_1<x_2$, then in $S'$ the event $P_2$ occurs first (with $v>0$).
Hence in general the simultaneity is frame-dependent.\\
The question of causality then arises, as different observers in different frames see different orders of events.
So we want to see a consistent ordering of cause and effect.
Note that the lines of simultaneity in $S'$ viewed in $S$ cannot incline more than $\pi/4$ since $|v|<c$.
In higher dimensions, lines and surfaces emerging from an event $P$ with $\pi/4$ inclination to the axes forms the light cones, the past light cone and the future light cone (depending on signs).
All observes agree that the event $Q$ occurs after $P$ if $Q$ is in the future light cone, but whether or not the event $R$, not in the light cones, occurs after $P$ is frame dependent.
The fact that $R$ is outside of the light cone of $P$ then implies that $R$ cannot be influenced by $P$, and vice versa, since matters cannot travel faster than the speed of light (which is the boundary of the light cone).
In general, an event can only be influenced by events in its past light cone and influence events in its future light cone.
So causality does preserve.\\
Now consider a clock stationary in $S'$ and tips in constant intervals $\delta t'$.
We want to know what time interval is perceived by observers in $S$.
Recall the inverse Lorentz transformation gives $t=\gamma(t'+x'v/c^2)$, but $x'$ is constant since the clock is stationary in $S'$.
So $\delta t=\gamma\delta t'$, so moving clocks are slower in moving frames.
This is called time dilation.
We say the time observed in the rest frame of a particular object the proper time.\\
Consider two twins, Luke and Leia.
Luke is staying home and Leia is going to a far planet and return home with speed $v$ relative to Luke.
In Luke's frame of reference, take the origin to be home.
Suppose the planet is at $x=P$ and Leia arrives at the planet at time $cT$.
Time experienced by Leia in this part of the journey is then
$$T'=\gamma(T-\frac{v}{c^2}vT)=\frac{T}{\gamma}$$
Same for going back.
So during the entire journey, Leia aged $2T/\gamma$ while Luke aged $2T$, so Leia becomes younger than Luke.
From Leia's perspective, Luke travels away from her and returns, so if the problem is symmetric, then Luke should be younger, which is a contradiction.
So the paradox is the lack of symmetries in this problem.
Let $X$ be the intersection point between the line of simultaneity in Leia's outward frame through $P$, so at $A$, we have $x=0,t=T,t'=T/\gamma$ and $X$ has $x=0,t'=T/\gamma$, so the time experienced by Leia would be $t=T/\gamma^2$ in Luke's frame at $A$.
As for the return journey, the line of simultaneity changes sign.
So in the return journey, Luke sees Leia aging from $A$ to $R$ and Leia sees Luke aging from $Z$ to $R$ (where $Z$ is the event with $x=0$ that is simultaneous with $A$ in the frame of Leia on her return journey).
The reason for the paradox is the discontinuity of time (from $X$ to $Z$) when Leia changes direction, so Luke has aged instaneously from $X$ to $Z$.\\
Now we shall talk about length contraction.
Consider a rod of length $L'$ stationary in $S'$, we want to know about the length of the rod in $S$.
Suppose the ends of the rod are at $x'=0$ and $x'=L'$, so the world lines of the ends are simply the two vertical lines described by these equations.
So $x'=0$ mapsto $\gamma(x-vt)=0$ in $S$ and $x'=L'$ mapsto $\gamma(x-vt)=L'$, so these two lines are still parallel but the horizontal (in $S$) distance between them are now $L=L'/\gamma$, so moving objects are contracted in the direction in which they move.
We define the proper length to be the length measured in the rest frame of the rod, which is essentially the greatest length of it over all frames.\\
A practical problem is that does a train of proper length $2L$ fits in a platform of proper length $L$ if it travels at a certain speed.
So we want $\gamma=2$.
Now for the observers at the platform, this would work if the train attains the desired speed.
As for the observers on the train, the platform contracts to length $L/\gamma=L/2$ so it doesn't fit.
Suppose the platform is defined by $x=0$ and $x=L$.
The train is defined by $x'=0$ and $x'=2L$, which are mapped to some slanted lines in $S$, the frame of the platform.
Consider the event $E$ where the rear of the platform and the rear of the train coincide.
For simplicity, this happens at $t=t'=0$.
Now the front of the train is $x'=2L$ and the platform is $x=L$.
Let $F$ be the event which is simultaneous with $E$ in $S$ at the front of the train, so $x'=\gamma(x-vt),2L=\gamma(L-vt)$ which implies $t=0$, so in the platform, $E$ is simultaneous with $F$, but in the train $S'$, we have $t'<0$ by calculation.
So in the train $F$ occurs before $E$ in $S'$.\\
Now that both length and time become different in different frames, what about velocities?
Suppose we have a particle moving with constant velocity $u'$ in $S'$ which moves with constant velocity $v$ relative to $S$.
We want to know the velocity $u$ of the particle as measured in $S$.
The world line of the particle in $S'$ can be taken as $x'=u't'$, so we have $\gamma(x-vt)=u'\gamma(t-vx/c^2)$ which gives $u=(u'+v)/(1+u'v/c^2)$.
In particular, if $u',v<<c$, then $u\approx u'+v$ which is the standard Galilean transformation.
Note also that we still cannot get to the speed of light given $u',v<c$ which is a combination of successive boosts.
\subsection{The Geometry of Space-time}
\begin{definition}
    Consider two points $P,Q$ in space-time having coordinates $(x_1,ct_1),(x_2,ct_2)$, so $\delta t=t_2-t_1$ and the space seperation is $\delta x=x_2-x_1$.
    We define the invariant interval between $P,Q$ to be $\delta s^2=c^2\delta t^2-\delta x^2$.
\end{definition}
Note that as we observed before, one can show that all observers agree on the value of $\delta s^2$.
\begin{definition}
    If we have three spacial dimensions $(x,y,z)$, we define $\delta s^2=c^2\delta t^2-\delta x^2-\delta y^2-\delta z^2$.
\end{definition}
If the seperation between $P,Q$ becomes small, then $\mathrm ds^2=c^2\,\mathrm dt^2-(\mathrm dx^2+\mathrm dy^2+\mathrm dz^2)$ which looks like a distance (no it doesn't).
We can (no we can't) say that space-time is topologically equivalent to $\mathbb R^4$ endowed by the distance measure $\delta s$, but note that this is not even positive definite.
The space-time endowed with this ``measure of distance'' is called the Minkowski space-time.
\begin{definition}
    Two events having $\delta s^2<0$ are said to be time-like seperated, and two that have $\delta s^2<0$ are said to be space-like seperated.
\end{definition}
So two time-like seperated events are at the same space position in some frame of reference and space-like seperated events are at the same time position in some frame.
\begin{definition}
    if $\delta s^2=0$, we say $P,Q$ are light-like seperated, so they can be connected by a light ray.
\end{definition}
Note also that events that are light-like seperated may not be the same.
\begin{definition}
    Take event $P$ in $S$, we can write its coordinates as a $4$-vector $X^\mu=(ct,x,y,z),\mu=0,1,2,3$, so $X^0=ct$ etc..
\end{definition}
We can define a new ``inner product'' on $4$-vectors by $X\cdot Y=X^\top\eta X=X^\mu\eta_{\mu\nu}X^\nu$ where
$$\eta=\begin{pmatrix}
    1&0&0&0\\
    0&-1&0&0\\
    0&0&-1&0\\
    0&0&0&-1
\end{pmatrix}$$
So we have $X\cdot X=c^2t^2-x^2-y^2-z^2$.
We call this the Minkowski metric.
$4$-vectors with $X\cdot X>0$ are time-like, those with $X\cdot X<0$ are space-like and those with $X\cdot X=0$ are light-like (or null).
The Lorentz transformation is a lineaR transformation that takes the components of a $4$-vector in $S$ to those of a $4$-vector in $S'$.
Hence we can write it as a matrix $\Lambda$ where $X'=\Lambda X$.
The set of all $\Lambda$ that preserves the Minkowski metric then forms a group, called the Lorentz group.
I.e. we want $X\cdot X=(\Lambda X)\cdot (\Lambda X)$ for all $4$-vector $X$.
By substitution we have $\Lambda^\top\eta\Lambda=\eta$.\\
If $\Lambda$ is just a spacial transformation, i.e.
$$\Lambda=\begin{pmatrix}
    1&0&0&0\\
    0&&&\\
    0&&R&\\
    0&&&
\end{pmatrix}$$
So $R$ must be a rotation.
If it is not the case, we can also have the boost (WLOG in the $x$-direction)
$$\Lambda=\begin{pmatrix}
    \gamma&-\gamma\beta&0&0\\
    -\gamma\beta&\gamma&0&0\\
    0&0&1&0\\
    0&0&0&1
\end{pmatrix},\beta=\frac{v}{c}$$
The Lorentz group $\operatorname{O}(1,3)$ also consists of spacial reflections and time reversals.
And its subgroup $\operatorname{SO}(1,3)$ with determinant $1$ is called the proper Lorentz group.
This includes composition of time reversals and spacial reflections.
The subgroup that preserves the direction of time and spacial orientation is called the restrictive Lorentz group $\operatorname{SO}^+(1,3)$, which is generated by spacial rotations and boosts (in all directions).\\
A way to label the Lorentz transformations is by a concept of rapidity.
We now focus on the $(ct,x)$ space (i.e. the $2\times 2$ submatrix on the top left corner operating on $(ct,x)$), where we define
$$\Lambda[\beta]=\begin{pmatrix}
    \gamma&-\gamma\beta\\
    -\gamma\beta&\gamma
\end{pmatrix}$$
So if we combine two boosts in the $x$ direction, then we have $\Lambda[\beta_1]\Lambda[\beta_2]=\Lambda[(\beta_1+\beta_2)/(1+\beta_1\beta_2)]$ with appropriate values of $\gamma$'s.
Recall that for spacial rotations, we have $R(\theta_1+\theta_2)=R(\theta_1)R(\theta_2)$.
For Lorentz boosts, we define the rapidity $\phi$ by $\beta=\tanh\phi$, so $\gamma=\cosh\phi$ and $\gamma\beta=\sinh\phi$.
Hence
$$\Lambda[\phi]=\begin{pmatrix}
    \cosh\phi&-\sinh\phi\\
    -\sinh\phi&\cosh\phi
\end{pmatrix}$$
and thus $\Lambda[\phi_1]\Lambda[\phi_2]=\Lambda[\phi_1+\phi_2]$.
This suggests that Lorentz transformations are hyperbolic rotations of space-time.
\subsection{Relativistic Kinematics}
Consider a particle moving along some trajectory $\underline{x}(t)$, then $\underline{u}(t)=\mathrm d\underline{x}/\mathrm dt$, so the path of it in space-time is parameterized by $t$.
But in special relativity the dependent variable $t$ is also going to change, so the path of it in a new frame would be non-trivial.
Consider a particle at rest in $S'$, so $\underline{x}=\underline{x_0}$ in $S'$, so the invariant interval would be $\delta s^2=c^2\delta t'^2$.
Define the proper time as the time $\tau$ with $c^2\delta\tau^2=\delta s^2$, so $\delta\tau$ is the time experienced by the particle.
Due to invariance, this equation holds in all frame, and $\tau$ is real in time-like intervals.
So the world line of a particle can be parameterized by $\tau$.
In terms of an infinitesimal interval, if $\underline{u}$ is the speed of the particle, we have
$$\mathrm d\tau=\frac{\mathrm ds}{c}=\frac{1}{c}\sqrt{c^2\,\mathrm dt^2-\mathrm dx^2}=\sqrt{1-\frac{|\underline{u}|^2}{c^2}}\mathrm dt$$
Hence $\mathrm dt/\mathrm d\tau=\gamma_u$ where $\gamma_u=1/\sqrt{1-u^2/c^2}$.
The total time experienced by the particle is then
$$T=\int\mathrm d\tau=\int\frac{\mathrm dt}{\gamma_u}$$
To study this, we introduce the concept of a $4$-velocity.
The position $4$-vector of a particle is the column vector $X(\tau)=(ct(\tau),\underline{x}(\tau))^\top$ where $\underline{x}$ is a $3$-vector.
\begin{definition}
    The $4$-velocity is
    $$U=\frac{\mathrm dX}{\mathrm d\tau}=\begin{pmatrix}
        c\,\mathrm dt/\mathrm d\tau\\
        \mathrm d\underline{x}/\mathrm d\tau
    \end{pmatrix}=\gamma_u\begin{pmatrix}
        c\\
        \underline{u}
    \end{pmatrix},\underline{u}=\frac{\mathrm d\underline{x}}{\mathrm dt}$$
\end{definition}
If I have two frames $S,S'$ such that the components of $X,X'$ of the position vector are related by $X'=\Lambda X$, then $U'=\Lambda U$.
In general, everything that transforms in this way is called a $4$-vector.
And in particular, $U$ is a $4$-vector since $X$ is and $\tau$ is invariant.
Note that $\mathrm dX/\mathrm dt$ is however not a $4$-vector.
The scalar product $U\cdot U$ will hence be Lorentz invariant.
That is, $U\cdot U=U'\cdot U'$.
In the rest frame where the particle with $4$-velocity $U$, then $U=(c,\underline{0})^\top$, so $U\cdot U=c^2$, so for any $u$, we have $c^2=\gamma_u^2(c^2-|\underline{u}|^2)$.
We have seen that the rule of transformation of velocity in special relativity is not as simple as in Galilean transformations.
However, we do have a fairly simple transformation law for $4$-vectors, which we can apply to $4$-velocity, which gives $U'=\Lambda U$.
\begin{example}
    In a frame $S$ where our favourite particle is moving with a speed $u$ at an angle $\theta$ to the $x$-axis in the $x-y$ plane.
    Its $4$-velocity is then $U=\gamma_u(c,u\cos\theta,u\sin\theta,0)^\top$.
    Consider another frame $S'$ which moves with speed $v$ in the $x$ direction of $S$.
    Suppose the velocity in $S$ is $u'$ and it makes an angle $\theta'$ to the $x$-direction in $S'$.
    So $U'=\gamma_{u'}(c,u'\cos\theta',u'\sin\theta',0)$ with
    $$U'=\begin{pmatrix}
        \gamma_v&-\gamma_v\beta_v&0&0\\
        -\gamma_v\beta_v&\gamma_v&0&0\\
        0&0&1&0\\
        0&0&0&1
    \end{pmatrix}U$$
    which we can solve to get $\theta',u'$ in terms of other things.
    $$u'\cos\theta'=\frac{u\cos\theta-v}{1-uv\cos\theta/c^2},\tan\theta'=\frac{u\sin\theta}{\gamma_v(u\cos\theta-v)}$$
    This change in angle, i.e. apparent change of direction, of the motion of the particle due to the motion of the observer is called an aberation.
    This also applied with the particle is a photon, so $u=c$, so although the speed of light cannot change across inertial frames, the direction of light ray can.
\end{example}
We also want to talk about $4$-momentum.
The $4$-momentum of a particle of mass $m$ moving with $4$-velocity $u$ is given by $P=mU=m\gamma_u(c,\underline{u})^\top$ with components $\mu=0,1,2,3$ where the component $\mu=0$ is interpreted as time.
For $P$ to be a $4$-vector, $m$ must be an invariant, so we must take $m$ to be the rest mass of the particle.
The $4$-momentum of a system of particles is the sum of the individual particles which conserves in the absence of external forces.
The spacial components of $P$ corresponds to the relativistic $3$-momentum $\underline{p}=\gamma_um\underline{u}$ which is the same as the Newtonian expression except that mass is modified to $\gamma_um$, which is interpreted as the apparent mass of the moving particle.
In particular, when $|\underline{u}|\to c$, the apparent mass tends to infinity.
For the zero component,
$$P^0=\gamma_umc=\frac{1}{c}\left(mc^2+\frac{1}{2}m|\underline{u}|^2+\cdots\right)$$
We see the kinetic energy in the second term, so the natural interpretation of $P$ is $P=(E/c,\underline{p})$ where $E$ is called the relativistic energy, so $E=\gamma_umc^2=mc^2+m|\underline{u}|^2/2+\cdots$ and $P$ is sometimes called the energy-momentum $4$-vector.
Note that $E\to\infty$ as $|\underline{u}|\to c$.
So for a stationary particle, we have $E=mc^2$, and for a moving particle we have $E=mc^2+(\gamma_u-1)mc^2$ where the second term is the kinetic energy, which reduces to the Newtonian kinetic energy for small $u$.
Now $P\cdot P=E^2/c^2-|\underline{p}|^2$ is conserved under Lorentz transformation and hence equals to $m^2c^2$, so we have $E^2=|\underline{p}|^2c^2+m^2c^4$.
In Newtonian physics, mass and energy are seperated idea in the sense that they are seperately conserved.
But in relativity, mass is not conserved and is a form of energy, i.e. we can convert mass into kinetic energy and vice versa.\\
Consider a massless particles (i.e. particles with zero rest mass) like a photon.
It can have non-zero ($4$-)momentum and hence nonzero relativistic energy.
Suppose this particle has the speed of light, then $0=m^2c^2=P\cdot P=E^2/c^2-|\underline{p}|^2$.
We say this particle is light-like and it travels through a light-like trajectory.
Consequently, there is no proper time for this particle.
Note that in this case $E=c|\underline{p}|$, so
$$P=\frac{E}{c}\begin{pmatrix}
    1\\
    \underline{n}
\end{pmatrix}$$
where $\underline{n}$ is a unit vector.
In special relativity, we can write Newton's Law as
$$\frac{\mathrm dP}{\mathrm d\tau}=F$$
where $F$ is the $4$-force, i.e.
$$F=\gamma_u\begin{pmatrix}
    \underline{F}\cdot\underline{u}/c\\
    \underline{F}
\end{pmatrix}$$
which is also a $4$-vector.
Note that if we transform from proper time to time experienced, then Newton's Second Law pops up, so it is consistent.
Equivalently, for a particle with rest mass $m$, then one can write $F=mA$ where $A=\mathrm dU/\mathrm d\tau$ is the $4$-acceleration.
\subsection{Examples in Particle Physics}
We want to explore the use of the conservation of total $4$-momentum in problems in particle physics.
Consider $P=(E/c,\underline{p})^\top$ for a system of particles.
A useful way to consider the system is to introduce the notion of a center-of-momentum frame (CM frame), which is the frame where the total $4$-momentum is $0$ (possible whenever all particles have positive rest mass).
\begin{example}
    Particle decay.
    Consider a particle of mass $m_1$ with momentum $P_1$ which is deemed to decay into two particles of mass $m_2,m_3$ and momenta $P_2,P_3$ respectively.
    So we have $P_1=P_2+P_3$.
    Consider the zero component, then $E_1=E_2+E_3$.
    Consider the spacial components gives $\underline{p_1}=\underline{p_2}+\underline{p_3}$.
    In the CM frame, $P_1=0$, therefore $P_2=-P_3$.
    Also
    $$m_1c=E_1/c=E_2/c+E_3/c=\sqrt{|\underline{p_2}|^2+m_2^2c^2}+\sqrt{|\underline{p_3}|^2+m_3^2c^2}\ge (m_2+m_3)c$$
    So this decay is possible only if $m_1\ge m_2+m_3$.
    Note that is possible that we don't have the equality (unlike in Newtonian mechanics) where some mass has been converted to energy.
\end{example}
\begin{example}
    A Higgs particle $h$ is decayed into two photons $\gamma$, then $P_h=P_{\gamma_1}+P_{\gamma_2}$, then in the rest frame of $h$, $P_h=(m_hc,\underline{0})$.
    So if we look at the spacial components, then $\underline{0}=\underline{P_{\gamma_1}}+\underline{P_{\gamma_2}}$.
    And since the photons have zero rest mass,
    $$\frac{E_{\gamma_1}}{c}=|\underline{p_{\gamma_1}}|=|\underline{p_{\gamma_2}}|=\frac{E_{\gamma_2}}{c}$$
    So each of the photons has half of the Higgs particle's total energy.
    Note that in this case mass does not conserve.
\end{example}
\begin{example}
    Consider two identical particles which collide and retain their identities.
    Let $P_1,P_2$ be the $4$-momenta before and $P_3,P_4$ after respectively.
    Suppose $S$ is the laboratory frame where $\underline{p_2}=0$, let the horizontal to be the line joining the two particles and let $\theta$ be the inclination of particle $1$ after the collision and $\phi$ be that of particle $2$.
    We want to study the relationship between $\theta$ and $\phi$.\\
    Now we go to the CM frame where the particles are horizontal before the collision, then the trajectories form two lines crossing each other.
    Let $\theta'$ be the angle between those two lines.
    Let $v$ be the speeds before the collision and $w$ be that after.
    We put a $'$ to indicate we are in the CM frame.
    Then
    $$P_1'=\begin{pmatrix}
        m\gamma_vc\\
        m\gamma_vv\\
        0\\
        0
    \end{pmatrix},P_2'=\begin{pmatrix}
        m\gamma_vc\\
        -m\gamma_vv\\
        0\\
        0
    \end{pmatrix}$$
    and
    $$P_3'=\begin{pmatrix}
        m\gamma_wc\\
        m\gamma_ww\cos\theta'\\
        m\gamma_ww\sin\theta'\\
        0
    \end{pmatrix},P_4'=\begin{pmatrix}
        m\gamma_wc\\
        -m\gamma_ww\cos\theta'\\
        -m\gamma_ww\sin\theta'\\
        0
    \end{pmatrix}$$
    The first component gives $v=w$.
    Now we apply the Lorentz transformation from the CM frame $S'$ back to $S$.
    The velocity of the transformation is $v$, so
    $$\Lambda=\begin{pmatrix}
        \gamma_v&\gamma_vv/c&0&0\\
        \gamma_vv/c&\gamma_v&0&0\\
        0&0&1&0\\
        0&0&0&1
    \end{pmatrix}$$
    Now before the collision,
    $$P_1=\begin{pmatrix}
        m\gamma_v^2(c+v^2/c)\\
        m\gamma_v^2(v+v)\\
        0\\
        0
    \end{pmatrix}=\begin{pmatrix}
        m\gamma_uc\\
        m\gamma_uu
    \end{pmatrix}$$
    where $u$ is the initial velocity of particle $1$.
    Consider the situation after the collision and set $q$ to be the velocity of particle $1$ after the collision, we get
    $$P_3=\begin{pmatrix}
        m\gamma_v^2(c+(v^2/c)\cos\theta')\\
        m\gamma_v^2(v+v\cos\theta')\\
        m\gamma_vv\sin\theta'\\
        0
    \end{pmatrix}=\begin{pmatrix}
        m\gamma_qc\\
        m\gamma_qq\cos\theta\\
        m\gamma_qq\sin\theta\\
        0
    \end{pmatrix}$$
    So by comparing the $1$ and $2$ components, we get
    $$\tan\theta=\frac{m\gamma_v}{m\gamma_v^2}\frac{v\sin\theta'}{v(1+\cos\theta')}=\frac{1}{\gamma_v}\tan\frac{\theta'}{2}$$
    Similarly,
    $$\tan\phi=\frac{1}{\gamma_v}\cot\frac{\theta'}{2}$$
    So $\tan\theta\tan\phi=1/\gamma_v^2=2/(1+\gamma_u)$.
    When $\gamma_u\to 1$ (i.e. in the Newtonian limit), we get $\tan\theta\tan\phi=1$, so the angle after the collision would be $\pi/2$.
\end{example}
\end{document}