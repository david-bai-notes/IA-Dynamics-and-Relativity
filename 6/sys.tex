\section{System of Particles}
We have considered the motion of a single particle in a force field, so we will now turn to a system of particles where they act on each other.\\
Consider $N$ particles, namely particles $1,\ldots,i,\ldots,N$ having masses $m_i$ and positions $\underline{r_i}(t)$.
The momentums are then $\underline{p_i}=m_i\underline{\dot{r}_i}$.
Newton's Second Law for one of these particles is then $m_i\underline{\ddot{r}_i}=\underline{\dot{p}_i}=\underline{F_i}$.
Divide the forces exerted in two parts: the external forces $\underline{F_i}^{\rm ext}$ (causes by something outside the $N$ particles) and internal forces $\underline{F_{ij}}$ which is the force exerted by particle $j$ on $i$.
Conventionally we take $\underline{F_{ii}}=\underline{0}$.
So basically
$$\underline{F_i}=\underline{F_i}^{\rm ext}+\sum_j\underline{F_{ij}}$$
Newton's Third Law then tells us $\underline{F_{ij}}+\underline{F_{ji}}=0$, like gravitation.
\subsection{Motion of the Centre of Mass}
The total mass of the system is $M=\sum_im_i$, then we define the centre of mass to be a location
$$\underline{R}=\frac{1}{N}\sum_{i=1}^Nm_i\underline{r_i}$$
The total linear momentum would be $\underline{P}=\sum_i\underline{p_i}=\sum_im_i\underline{\dot{r}}=M\underline{\dot{R}}$.
Consider the rate of change of the momentum.
By Newton's Secon Law,
$$\underline{\dot{P}}=M\underline{\dot{R}}=\sum_{i=1}^N\underline{\dot{p}_i}=\sum_{i=1}^N\underline{F_i}^{\rm ext}+\sum_{i=1}^N\sum_{j=1}^N\underline{F_{ij}}=\sum_{i=1}^N\underline{F_i}^{\rm ext}=\underline{F}^{\rm ext}$$
By $\underline{F_{ij}}=-\underline{F_{ji}}$.
So the motions of the centre of mass closely resembles that of a single particle with mass $M$ and position $\underline{R}$, which is reassuring since it means we can consider finite (bounded) bodies as particles.
So Newton's Second Law applies to macroscopic objects.
A simple conclusion from this is if $\underline{F}^{\rm ext}=0$, then the total momentum of the system $\underline{\dot{P}_i}=\sum_{i=1}^N\underline{\dot{p}_i}$ is conserved.
So in this case, we can set up the inertial frame as the centre of mass frame, in which $\underline{\dot{R}}=\underline{0}$.\\
Now we turn to consider the angular momentum.
Consider the total angular momentum about the origin, which is $\underline{L}=\sum_i\underline{r_i}\times\underline{p_i}$, then
\begin{align*}
    \underline{\dot{L}}&=\sum_{i=1}^N\underline{r_i}\times\underline{\dot{p}_i}+\sum_{i=1}^N\underline{\dot{r}_i}\times\underline{p_i}\\
    &=\sum_{i=1}^N\underline{r_i}\times\underline{\dot{p}_i}\\
    &=\sum_{i=1}^N\underline{r_i}\times\underline{F_i}^{\rm ext}+\sum_{i=1}^N\sum_{j=1}^N\underline{r_i}\times\underline{F_{ij}}\\
    &=\underline{G}^{\rm ext}+\frac{1}{2}\sum_{i=1}^N\sum_{j=1}^N(\underline{r_i}-\underline{r_j})\times\underline{F_{ij}}
\end{align*}
The last term is sometimes zero, in which case the rate of change would be $\underline{G}^{\rm ext}$, the total external torque.
\subsection{Motion relative to the Centre of Mass}
Write $\underline{r_i}=\underline{R}+\underline{s_i}$, so $\underline{s_i}$ is the position of particle $i$ relative to the centre of mass.
Then
$$\sum_{i=1}^Nm_i\underline{s_i}=\sum_{i=1}^Nm_i(\underline{r_1}-\underline{R})=\sum_{i=1}^Nm_i\underline{r_i}-M\underline{R}=0$$
Consequently $\sum_im_i\underline{\dot{s}_i}=0$.
As for the total linear momentum, since we have the above,
$$\underline{P}=\sum_{i=1}^Nm_i(\underline{\dot{R}}+\underline{\dot{s}_i})=M\underline{\dot{R}}$$
The angular momentum would have, exploiting the same fact,
\begin{align*}
    \underline{L}&=\sum_{i=1}^Nm_i(\underline{R}+\underline{s_i})\times(\underline{\dot{R}}+\underline{\dot{s}_i})\\
    &=\sum_{i=1}^Nm_i\underline{R}\times\underline{\dot{R}}+\sum_{i=1}^Nm_i\underline{s_i}\times\underline{\dot{s}_i}+\left( \sum_{i=1}^Nm_i\underline{s_i} \right)\times\underline{\dot{R}}+\underline{R}\times\left( \sum_{i=1}^Nm_i\underline{\dot{s}_i} \right)\\
    &=\sum_{i=1}^Nm_i\underline{R}\times\underline{\dot{R}}+\sum_{i=1}^Nm_i\underline{s_i}\times\underline{\dot{s}_i}\\
    &=M\underline{R}\times\underline{\dot{R}}+\sum_{i=1}^Nm_i\underline{s_i}\times\underline{\dot{s}_i}
\end{align*}
which is the angular momentum of the centre of mass plus the total angular momentum relative to the centre of mass.
The total kinetic energy is
\begin{align*}
    T&=\sum_{i=1}^N\frac{1}{2}m_i|\underline{\dot{r}_i}|^2\\
    &=\sum_{i=1}^N\frac{1}{2}m_i|\underline{\dot{R}}+\underline{\dot{s}_i}|^2\\
    &=\sum_{i=1}^N\frac{1}{2}m_i|\underline{\dot{R}}|^2+\sum_{i=1}^N\frac{1}{2}m_i|\underline{\dot{s}_i}|^2+\underline{\dot{R}}\cdot\left(\sum_{i=1}^Nm_i\underline{\dot{s}_i}\right)\\
    &=\frac{1}{2}M|\underline{\dot{R}}|^2+\frac{1}{2}\sum_{i=1}^Nm_i|\underline{\dot{s}_i}|^2
\end{align*}
Now is the energy conserved?
Assuming $\underline{F}^{\rm ext}$ is conserved, then $\underline{F}^{\rm ext}=-\nabla V^{\rm ext}$.
Take $\underline{F_{ij}}$ to be conserved as well and its potential purely depends on the seperation between the particles, then $\underline{F_{ij}}=-\nabla V_{ij}(\underline{r_i}-\underline{r_j})$.
We can show that the total energy is conserved under these assuming these by just differentiating.
\subsection{The Two Body Problem}
The centre of mass is $\underline{R}=(m_1\underline{r_1}+m_2\underline{r_2})/M$
Consider the seperation vector $\underline{r}=\underline{r_1}-\underline{r_2}$, so we can write
$$\begin{cases}
    \underline{r_1}=\underline{R}+\frac{m_2}{m}\underline{r}\\
    \underline{r_2}=\underline{R}-\frac{m_1}{m}\underline{r}
\end{cases}$$
Since the external force is assumed to be zero, $\underline{R}$ moves with constant velocity.
Consider $\ddot{r}$, then $\ddot{r}=\underline{F_{12}}/m_1-\underline{F_{21}}/m_2=(1/m_1+1/m_2)\underline{F_{12}}$.
Hence
$$\mu\underline{\ddot{r}}=\underline{F_{12}}(\underline{r}),\mu=\frac{m_1m_2}{m_1+m_2}$$
Here $\mu$ is called the reduced mass.
In the case of the gravitational force, we have
$$\mu\underline{\ddot{r}}=-\frac{Gm_1m_2}{|\underline{r}|^3}\underline{r}\implies\underline{\ddot{r}}=-G(m_1+m_2)\frac{\underline{r}}{|\underline{r}|^3}=-\frac{GM}{|\underline{r}|^3}\underline{r}$$
which is just like the motion of two particles entirely due to a mass $M$ at the origin.
So both masses perform orbits with similar shape but different sizes.
\subsection{Variable Mass Problem}
Think of a rocket whose mass decreases as it moves due to the exhausted mass.
So its mass itself is variable.
So we need to apply Newton's Second Law to the whole system including the exhausted mass.
Suppose the mass and the velocity of the rocket is $m(t),v(t)$.
And the exhausted mass has a speed $u$ relative to the rocket when leaving the rocket.
At time $t$, when we look at this instant only, we can ignore what happened in the past, hence after a small time interval $\delta t$, mass of $m(t)=m(t+\delta t)$ is exhausted with speed $v(t)-u+o(\delta t)$.
By conservation of momentum,
$$m(t+\delta t)v(t+\delta t)+(m(t)-m(t+\delta t))(v(t)-u+o(\delta t))=m(t)v(t)$$
Hence
$$(mv^\prime+m^\prime u)\delta t\approx m(t+\delta t)(v(t+\delta t)-v(t))+(m(t+\delta t)-m(t))u+o(\delta t)=0$$
Since $\underline{F}=\underline{\dot{p}}$, it generalises to
$$mv^\prime+m^\prime u=F$$
where $F$ is the total external force exerted on the rocket.
This is called the rocket equation.
If $F=0$, we have $mv^\prime+m^\prime u=0$ which we can solve to get $v=v_0+u\log(m_0/m(t))$.